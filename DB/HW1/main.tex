\documentclass[12pt,a4paper]{article}
\usepackage[margin=2cm]{geometry}
\usepackage{xeCJK}
\usepackage{fontspec}
\setCJKmainfont{Noto Serif CJK TC}[Script=CJK]
\usepackage{amsmath,amssymb}
\usepackage{graphicx}
\usepackage{fancyhdr}
\setlength{\headheight}{14.5pt}
\addtolength{\topmargin}{-2.5pt}
\usepackage{hyperref}
\usepackage{listings}
\usepackage{enumitem}
\usepackage{titlesec}
\usepackage{caption}
\usepackage{indentfirst}
\usepackage{float}
\usepackage{forest}
\setlength{\parindent}{2em}
\pagestyle{fancy}
\fancyhf{}
\cfoot{\thepage}
\linespread{1.3}


\usepackage{minted}
\setminted{
    linenos,                % 行號
    frame=lines,            % 上下框線
    framesep=5pt,           % 程式碼與邊框距離
    numbersep=8pt,          % 行號與程式碼距離
    fontsize=\scriptsize,   % 字體大小
    breaklines,             % 自動換行
    tabsize=4,              % tab 寬度
    rulecolor=\color{black},% 框線顏色
    xleftmargin=1.5em       % 左側縮排
}


\title{資料庫管理 HW01}
\author{B12508026戴偉璿}
\date{\today}

\begin{document}

\maketitle

\lhead{資料庫管理 HW01}
\rhead{B12508026戴偉璿}
\newpage

\begin{enumerate}
    \item
    \begin{enumerate}
        \item 使用 python 統計結果如下:
        \begin{table}[H]
        \centering
        \begin{tabular}{c r r}
        \hline
        workday & registered & casual \\
        \hline
        0 & 683,537 & 316,732 \\
        1 & 1,989,125 & 303,285 \\
        \hline
        \end{tabular}
        \caption{依工作日與否統計會員與非會員的總租借數量}
        \label{tab:tab 1}
        \end{table}
        由此表格可得知,會員在工作日的需求量遠大於非工作日;非會員在非工作日的需求量則略高於工作日。但考量到可能有極端數據的影響,我也同時也統計了「會員在上班日租借大於非上班日之平均/中位數」、「非會員在非上班日租借大於上班日之平均/中位數」的天數: (上班日 500 天,非上班日 231 天)
        \begin{table}[H]
        \centering
        \begin{tabular}{c c c}
        \hline
        CMP base & Registered & Casual \\
        & (workday > non-workday) & (non-workday > workday) \\
        \hline
        Mean   & 379(75.8\%) & 173(74.8\%) \\
        Median & 381(76.2\%) & 172(74.5\%) \\
        \hline
        \end{tabular}
        \caption{使用平均數與中位數比較假說成立的天數}
        \end{table}
        由此表格可得知,無論是使用平均數或中位數來比較,皆有約七成五的天數符合假說,綜合Table \ref{tab:tab 1}的結果,因此「會員需求在上班日比非上班日高、非會員需求在非上班日比上班日高」假說成立。

        以下是程式碼:

        \begin{minted}{python}
            import pandas as pd

            data = pd.read_csv('Bike.csv')

            summary = data.groupby("workday")[["registered", "casual"]].sum()
            print(summary)

            print("sum of workday", (data["workday"] == 1).sum())
            print("sum of non-workday", (data["workday"] == 0).sum())

            # not a workday, registered mean
            avg_registered_nonwork = data.loc[data["workday"]==0, "registered"].mean()

            # not a workday, registered median
            med_registered_nonwork = data.loc[data["workday"]==0, "registered"].median()

            # a workday, casual mean
            avg_casual_work = data.loc[data["workday"]==1, "casual"].mean()

            # a workday, casual median
            med_casual_work = data.loc[data["workday"]==1, "casual"].median()

            # cmp by mean
            registered_support_days = (data.loc[data["workday"]==1, "registered"] > avg_registered_nonwork).sum()
            casual_support_days = (data.loc[data["workday"]==0, "casual"] > avg_casual_work).sum()

            print("sum of registered support work > no work: ", registered_support_days)
            print("sum of casual support no work > work: ", casual_support_days)

            # cmp by median
            registered_support_days = (data.loc[data["workday"]==1, "registered"] > med_registered_nonwork).sum()
            casual_support_days = (data.loc[data["workday"]==0, "casual"] > med_casual_work).sum()
            print("sum of registered support work > no work: ", registered_support_days)
            print("sum of casual support no work > work: ", casual_support_days)
        \end{minted}

        \item 使用 python 統計結果如下:
        \begin{table}[H]
        \centering
        \begin{tabular}{c r r || c r r}
        \hline
        Month & Registered & Casual & Month & Registered & Casual \\
        \hline
        1  & 122{,}891 & 12{,}042 & 7  & 266{,}791 & 78{,}157 \\
        2  & 136{,}389 & 14{,}963 & 8  & 279{,}155 & 72{,}039 \\
        3  & 184{,}476 & 44{,}444 & 9  & 275{,}668 & 70{,}323 \\
        4  & 208{,}292 & 60{,}802 & 10 & 262{,}592 & 59{,}760 \\
        5  & 256{,}401 & 75{,}285 & 11 & 218{,}228 & 36{,}603 \\
        6  & 272{,}436 & 73{,}906 & 12 & 189{,}343 & 21{,}693 \\
        \hline
        \end{tabular}
        \caption{每月會員與非會員租借總數比較}
        \label{tab:monthly_demand}
        \end{table}
        由統計結果可知,會員在每個月的租借量皆大於非會員,因此「不論哪一個月,會員總需求總是比非會員總需求高」假說成立。

        以下是程式碼:

        \begin{minted}{python}
        import pandas as pd

        data = pd.read_csv('Bike.csv')

        monthly = data.groupby("month")[["registered", "casual"]].sum()

        print(monthly)
        \end{minted}

        \item 使用 python 統計,2011 年標準差約為 1378.75;2012 年則是1788.67,因此「第二年的需求變異程度高於第一年」假說成立。

        以下是程式碼:

        \begin{minted}{python}
        import pandas as pd

        data = pd.read_csv('Bike.csv')

        print("std of 2011 : ", data.loc[data["year"]==2011, "cnt"].std())
        print("std of 2012 : ", data.loc[data["year"]==2012, "cnt"].std())
        \end{minted}    
    \end{enumerate}
    \item
    \begin{enumerate}
        \item 就複雜度分析的層面,在比較字串時最糟糕的狀況皆是比較到最後一個字元,因此交換指標相較於交換字串而言並不會有複雜度層面的改善。但就實際操作層面,每次交換指標時僅須交換指標的值,而不需要實際複製字串內容,這樣可以減少不必要的記憶體操作,提高效率。
        \item 答案如下:
        
        \begin{forest}
        for tree={
        circle,draw,minimum size=8mm,inner sep=1pt,
        s sep=8mm, l sep=10mm
        }
        [35
        [27
            [16
            [1][2]
            ]
            [19
            [13][13]
            ]
        ]
        [20
            [18
            [4][5]
            ]
            [7]
        ]
        ]
        \end{forest}
        \item 可以。在一個樹狀結構中,每一個節點會有至多一個唯一的「最左子節點」及「最近右兄弟節點」;此外,在一個 Left-child-right-sibling 表示法的二元樹狀結構中,每個節點的「左子節點」以及「右子節點」也是唯一的,因此必然能在此結構中找到對應的樹狀結構的節點。此時,二元樹中的「左子節點」對應到樹狀結構中的「最左子節點」,而二元樹中的「右子節點」則對應到樹狀結構中的「最近右兄弟節點」。
    \end{enumerate}
    \item
    \begin{enumerate}
        \item 作業系統通常包含了:程序與執行序、CPU排程、記憶體管理、檔案系統、I/O系統管理、網路管理、安全性與權限管理等模組。
        \item 「哲學家用餐問題」是指,有幾位哲學家坐在圓桌要吃飯,每兩個人之間放一隻筷子,要同時取得兩隻才能就餐。哲學家們可能會「思考」或是「就餐」,當他們選擇就餐時就會去拿筷子。但由於筷子的數量有限,因此可能會發生「deadlock」的情況:每個哲學家都拿到一隻筷子,等不到另一隻。這個問題的本質是資源分配,不同程序會競爭有限的資源,若沒有適當分配,可能會導致系統無法繼續運作。其中一種解法為「服務生解法」,亦即在桌邊放一個服務生,當哲學家要拿筷子時必須先向服務生取得許可,服務生會確保不會讓所有哲學家同時拿到一隻筷子,這樣就能避免死結的發生。這個解法的概念和通訊協定中的令牌傳遞(Token Passing)類似,都是透過一個中介來控制資源的分配,確保系統能夠順利運作。
        \item 連續分配的原則是,在 HDD 中,一個檔案的資料儲存在連續的區域中。優點是在讀取時指針只要尋找一次起始位置,可以加快 I/O 速度以及減少磨損;此外,將資料放在連續的位置也可以支援同一個檔案中隨機存取,因為檔案是連續存放,起始位置加上偏移量就能直接找到目標區塊。但缺點也很顯然,長時間使用後,雖然總空間足夠,但可能沒有「足夠大的連續空間」,導致檔案無法配置。而且如果檔案會成長,事先要預留額外空間,否則超出時可能需要整個搬移到另一個更大的連續區塊,這使空間利用率不佳。在 SSD 中,檔案不需要連續存放也可以快速的查找,連續分配失去其優勢但劣勢依然存在,因此就沒必要連續分配了。
        \item 虛擬記憶體的目的是讓每個程式都能使用一個獨立的、連續的邏輯位址空間,而不需要完全受限於實體記憶體的大小。當程式切換或同時執行多個程式時,作業系統不必整個搬移程序,只需在需要的時候將對應的頁面從磁碟的 swap 區載入到記憶體;不活躍的頁面則可先換出。如此即可有效支援多工,並提升記憶體利用率。
    \end{enumerate}
\end{enumerate}

\end{document}