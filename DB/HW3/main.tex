\documentclass[12pt,a4paper]{article}
\usepackage[margin=2cm]{geometry}
\usepackage{xeCJK}
\usepackage{fontspec}
\setCJKmainfont{Noto Serif CJK TC}[Script=CJK]
\usepackage{amsmath,amssymb}
\usepackage{graphicx}
\usepackage{fancyhdr}
\setlength{\headheight}{14.5pt}
\addtolength{\topmargin}{-2.5pt}
\usepackage{hyperref}
\usepackage{listings}
\usepackage{enumitem}
\usepackage{titlesec}
\usepackage{caption}
\usepackage{indentfirst}
\usepackage{float}
\usepackage{forest}
\setlength{\parindent}{2em}
\pagestyle{fancy}
\fancyhf{}
\cfoot{\thepage}
\linespread{1.3}

\usepackage{multirow}
\usepackage{booktabs}   % 放在 preamble
\usepackage{graphicx}
\usepackage{afterpage}

% tikz tools for ER diagram
\usepackage{tikz}
\usetikzlibrary{shapes,positioning,calc}
\colorlet{lightgray}{gray!20}


\usepackage{minted}
\setminted{
    linenos,                % 行號
    frame=lines,            % 上下框線
    framesep=5pt,           % 程式碼與邊框距離
    numbersep=8pt,          % 行號與程式碼距離
    fontsize=\scriptsize,   % 字體大小
    breaklines,             % 自動換行
    tabsize=4,              % tab 寬度
    rulecolor=\color{black},% 框線顏色
    xleftmargin=1.5em       % 左側縮排
}


\title{資料庫管理 HW03}
\author{B12508026戴偉璿}
\date{}

\begin{document}

\maketitle

\lhead{資料庫管理 HW03}
\rhead{B12508026戴偉璿}

\begin{enumerate}
    \item 
    \begin{enumerate}
        \item Left join all advisors(e) and their advisees(s), if someone has no advisee, then s would be NULL.
        \begin{minted}{sql}
select e.id, e.name from employee as e
left join employee s on e.id=s.supervisor_id
where s.supervisor_id is null;
        \end{minted}
        \item Find the latest store id of each employee before 2025-01-05 and left join to the employee table.
        \begin{minted}{sql}
select e.id as employee_id, h.store_id
from employee e
left join employee_store_history h
on e.id = h.employee_id
and h.start_date_time=(
    select max(h2.start_date_time)
    from employee_store_history h2
    where h2.employee_id = e.id
        and h2.start_date_time <= '2025-01-05'
    );           
        \end{minted}
        \item Using \texttt{limit 1} to obtain the first store id and \texttt{limit 1 offset 1} to obtain the second store id (after skipping the first one), then join them to produce the final result.
        \begin{minted}{sql}
select e.id as employee_id,
    (select h1.store_id from employee_store_history h1 where h1.employee_id=e.id order by h1.start_date_time limit 1) as first_store_id,
    (select h2.store_id from employee_store_history h2 where h2.employee_id=e.id order by h2.start_date_time limit 1 offset 1) as second_store_id
from employee e;
        \end{minted}
    \end{enumerate}
\end{enumerate}

\end{document}