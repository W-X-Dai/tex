\documentclass[xcolor=dvipsnames]{beamer}

% ==== 主題 ====
\usetheme{metropolis}
\usefonttheme{professionalfonts}          % 不覆蓋你自訂的字型

% ==== 字型 ====
\usepackage{fontspec}
\usepackage{xeCJK}
\renewcommand{\familydefault}{\rmdefault} % 使用 serif 字體(重點)

% 西文字型:Times New Roman 的開源替代品
\setmainfont{TeX Gyre Termes}[
  Ligatures=TeX,
  BoldFont={* Bold},
  ItalicFont={* Italic}
]

% 中文字型(可改為思源宋體、標楷體等)
\setCJKmainfont{Noto Serif CJK TC}
\setCJKsansfont{Noto Sans CJK TC} % 有需要再用
\setCJKmonofont{Noto Sans Mono CJK TC}

% ==== 數學字型(與正文字體一致)====
\usepackage{unicode-math}
\setmathfont{TeX Gyre Termes Math}

% ==== 套件 ====
\usepackage{amsmath, amssymb}
\usepackage{graphicx}
\usepackage{hyperref}
\usepackage{minted}
\usepackage{fvextra}
\usepackage{xcolor}
\usepackage{booktabs}

% ==== 顏色設定(可選)====
\definecolor{MyBlue}{RGB}{3, 55, 105}
\setbeamercolor{structure}{fg=MyBlue}
\setbeamercolor{block title}{bg=MyBlue,fg=white}
\setbeamercolor{block body}{bg=blue!5}
\setbeamertemplate{section in toc}{%
  \inserttocsectionnumber.~\inserttocsection\par
}

\setminted{
    linenos,                % 行號
    frame=lines,            % 上下框線
    framesep=5pt,           % 程式碼與邊框距離
    numbersep=8pt,          % 行號與程式碼距離
    fontsize=\scriptsize,   % 字體大小
    breaklines,             % 自動換行
    tabsize=4,              % tab 寬度
    rulecolor=\color{black},% 框線顏色
    xleftmargin=1.5em       % 左側縮排
}

\title{Variable Operations}
\author{Tai, Wei Hsuan}
\date{week 2}

\begin{document}
	\begin{frame}
		\titlepage
	\end{frame}


    % course material qrcode
    \begin{frame}
        \frametitle{\href{https://drive.google.com/drive/folders/14Tkn-rddw0k1obeOxkWi00S43M0e9wlW?usp=sharing}{課程簡報}}
        \begin{figure}
            \centering
            \includegraphics[width=0.7\textwidth]{src/qrcode.png}
        \end{figure}
    \end{frame}

    \begin{frame}
        \frametitle{Outline}
        \tableofcontents
    \end{frame}

    \section{Recap}

    \begin{frame}[fragile]
        \frametitle{Basic Structure of C++ Program}
        \begin{itemize}
            \item Header file.
            \item Namespace.
            \item Main function.
        \end{itemize}
        \begin{minted}{cpp}
            #include <iostream>
            using namespace std;

            int main() {
                cout << "Hello, World!" << endl;
                return 0;
            }
        \end{minted}
    \end{frame}

    \begin{frame}
        \frametitle{Remember!}
        \begin{itemize}
            \item Every statement ends with a semicolon \textbf{(;)}.
            \item Programs always start from the \alert{main()} function.
            \item All functional code needs to be inside a function.
            \item All functions must be enclosed in curly braces \textbf{\{\}}.
        \end{itemize}
    \end{frame}

    \section{Variable Introduction}
    \begin{frame}
        \frametitle{What is Variable?}
        Variables can be thought of as \alert{containers} that hold data which can be changed during program execution.\\
        Data is actually stored in the computer's memory, and variables provide a way to access and manipulate that data.\\
        You can consider variables as \alert{labels} for specific memory locations where data is stored.
    \end{frame}

    \begin{frame}
        \frametitle{Common Data Types}
    \end{frame}


\end{document}