\documentclass[xcolor=dvipsnames]{beamer}

% ==== 主題 ====
\usetheme{metropolis}
\usefonttheme{professionalfonts}          % 不覆蓋你自訂的字型

% ==== 字型 ====
\usepackage{fontspec}
\usepackage{xeCJK}
\renewcommand{\familydefault}{\rmdefault} % 使用 serif 字體(重點)

% 西文字型:Times New Roman 的開源替代品
\setmainfont{TeX Gyre Termes}[
  Ligatures=TeX,
  BoldFont={* Bold},
  ItalicFont={* Italic}
]

% 中文字型(可改為思源宋體、標楷體等)
\setCJKmainfont{Noto Serif CJK TC}
\setCJKsansfont{Noto Sans CJK TC} % 有需要再用
\setCJKmonofont{Noto Sans Mono CJK TC}

% ==== 數學字型(與正文字體一致)====
\usepackage{unicode-math}
\setmathfont{TeX Gyre Termes Math}

% ==== 套件 ====
\usepackage{amsmath, amssymb}
\usepackage{graphicx}
\usepackage{hyperref}
\usepackage{minted}
\usepackage{fvextra}
\usepackage{xcolor}
\usepackage{booktabs}

% ==== 顏色設定(可選)====
\definecolor{MyBlue}{RGB}{3, 55, 105}
\setbeamercolor{structure}{fg=MyBlue}
\setbeamercolor{block title}{bg=MyBlue,fg=white}
\setbeamercolor{block body}{bg=blue!5}

\setminted{
    linenos,                % 行號
    frame=lines,            % 上下框線
    framesep=5pt,           % 程式碼與邊框距離
    numbersep=8pt,          % 行號與程式碼距離
    fontsize=\scriptsize,   % 字體大小
    breaklines,             % 自動換行
    tabsize=4,              % tab 寬度
    rulecolor=\color{black},% 框線顏色
    xleftmargin=1.5em       % 左側縮排
}

\title{Introduction to C++}
\author{Tai, Wei Hsuan}
\date{week 1}

\begin{document}
	\begin{frame}
		\titlepage
	\end{frame}

    \section{課程介紹}
    \begin{frame}
        \frametitle{關於我...}
        \begin{itemize}
            \item 戴偉璿(Tai, Wei Hsuan)
            \item 台大醫工大三
            \item 擅長:C++、資料結構與演算法、網際網路概論、網站前後端架設、Linux、機器學習...
            \item 興趣:寫程式、看棒球、打電動
            \item 張學友的粉絲
        \end{itemize}
    \end{frame}

    \begin{frame}
        \frametitle{進度安排}
        \begin{table}[h]
        \begin{tabular}{cl  cl}
        \toprule
        日期 & 主題 & 日期 & 主題 \\
        \midrule
        9/12  & 課程簡介、基礎輸入輸出   & 11/28 & 函式 \\
        9/19  & 變數與四則運算           & 12/5  & 遞迴 \\
        10/17 & 選擇結構與邏輯運算子     & 12/12 & Struct \\
        10/31 & 重複結構                 & 12/19 & Vector \\
        11/7  & 字串處理                 & 12/26 & Stack, Queue \\
        11/21 & 期中考                   & 1/2   & Set, Map, Priority Queue \\
            &                          & 1/9   & 期末考 \\
        \bottomrule
        \end{tabular}
        \end{table}
    \end{frame}

    \begin{frame}
        \frametitle{上課方式}
        \begin{itemize}
            \item 觀念講解、範例示範
            \item 大量的練習題
            \item 大量的數學證明(如果有必要)
        \end{itemize}
    \end{frame}

    \begin{frame}
        \frametitle{關於LLM}
        \begin{itemize}
            \item Large Language Model
            \item ChatGPT、Grok、Claude、Gemini
            \item 以討論取代抄答案
            \item \textbf{請不要叫我檢查你用LLM寫的程式碼!}
        \end{itemize}
    \end{frame}

    \section{Why C++?}
    
    \begin{frame}
        \begin{figure}
            \centering
            \includegraphics[width=1\textwidth]{src/why_not_App.png}
        \end{figure}
    \end{frame}

    % C++ vs Python
    \begin{frame}
        \frametitle{C++ vs Python}
        \begin{itemize}
            \item C++:
            \begin{itemize}
                \item Compiled Language
                \item Static Typing
                \item High Performance
                \item Working Close to Hardware
            \end{itemize}
            \item Python:
            \begin{itemize}
                \item Interpreted Language
                \item Dynamic Typing
                \item Easy to Learn and Use
                \item Rich Libraries and Frameworks
            \end{itemize}
        \end{itemize}
    \end{frame}

    % C++ 就藏在生活中
    \begin{frame}
        \frametitle{C++ 就藏在生活中!}
        \begin{itemize}
            \item Operating Systems(Windows、macOS、Linux)
            \item Web Browsers(Chrome、Firefox、Edge)
            \item Game Engines(Unreal Engine、Unity)
            \item Parallel Computing(CUDA、OpenCL)
        \end{itemize}
    \end{frame}

    \section{What is C++?}
    
    %Intro to C++
    \begin{frame}
        \frametitle{Introduction to C++}
        \begin{itemize}
            \item First released in Nokia Bell Labs by Bjarne Stroustrup
            \item C with Classes
            \item 1983s, C with Classes $\rightarrow$ C++
            \item Object-Oriented Programming
            \item Directly manipulate hardware resources(memory, CPU)
        \end{itemize}
    \end{frame}

    \begin{frame}
        \frametitle{"Father" of Programming Languages}
        \begin{itemize}
            \item CPython, Numpy, Pandas, Matplotlib, Scikit-learn, TensorFlow, PyTorch
            \item JavaScript V8 Engine, Node.js, Deno
            \item MySQL, PostgreSQL, MongoDB
            \item Even C++ compiler itself!
        \end{itemize}
    \end{frame}

    \begin{frame}
        \frametitle{Simple C++ IDE}
        \begin{itemize}
            \item Windows: \href{https://sourceforge.net/projects/orwelldevcpp/}{Dev-C++}
            \item macOS: \href{https://developer.apple.com/xcode/}{Xcode}
        \end{itemize}
    \end{frame}

    \section{Standard I/O in C++}

    \begin{frame}[fragile]
        \frametitle{\href{https://zh.wikipedia.org/zh-tw/Hello_Worldtext}{Hello World!}}
        \begin{minted}{cpp}
        #include <iostream>
        using namespace std;

        int main() {
            cout << "Hello World!" << endl;
            return 0;
        }
        \end{minted}
    \end{frame}

    \begin{frame}
        \frametitle{Basic Structure of C++ Program}
        \begin{itemize}
            \item Header Files
            \item Namespaces
            \item Main Function
        \end{itemize}
    \end{frame}

    \begin{frame}
        \frametitle{Header Files}
        In python, we use \texttt{import} to include libraries.\\
        In C++, we use \texttt{\#include} to include header files.\\
        The most common header file is \texttt{<iostream>}, which is used for input and output operations(io refers to input/output). But if you want to use other advanced features, you may need to include other header files(i.e. \texttt{<vector>}, \texttt{<algorithm>}). Therefore, we often use \texttt{\#include <bits/stdc++.h>} to include most standard libraries in contests(Do not do this in production code!).

    \end{frame}

    \begin{frame}
        \frametitle{Namespaces}
    \end{frame}


\end{document}

1. Why C++?
2. C++ 的基礎輸入輸出
3. 練習題