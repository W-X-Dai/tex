\documentclass[11pt,a4paper]{article}

\usepackage{indentfirst}
\usepackage{amssymb}
\usepackage{subcaption}
\usepackage{graphicx}
\usepackage{longtable}
\usepackage{fancyhdr}
\usepackage{xeCJK}
\usepackage{amsmath}
\usepackage{amssymb}
\usepackage{ulem}
\usepackage{xcolor}
\usepackage{fancyvrb}
\usepackage{listings}
\usepackage{soul}
\usepackage{hyperref}
	\lstdefinestyle{C}{
   		language=C, 
   		basicstyle=\ttfamily\bfseries,
    	numbers=left, 
    	numbersep=5pt,
    	tabsize=4,
    	frame=single,
   	 	commentstyle=\itshape\color{brown},
    	keywordstyle=\bfseries\color{blue},
   	 	deletekeywords={define},
    	morekeywords={NULL,bool}
	}	

\setCJKmainfont{Noto Serif CJK TC}[Script=CJK]
\setCJKmonofont{Noto Sans Mono CJK TC}[Script=CJK]
 
\voffset -20pt
\textwidth 410pt
\textheight 650pt
\oddsidemargin 20pt
\newcommand{\XOR}{\otimes}
\linespread{1.2}\selectfont

\pagestyle{fancy}
\lhead{114APCS進階班期中比賽}

\begin{document}

\begin{center}
\section*{A. Sum}
\end{center}

\section*{Description}

相信優秀的各位一定都聽過費波納契數列(義大利語:Successione di Fibonacci),這是一個很神奇的東西,有超級多的奇怪性質,比如說連續兩項前項與後項的比值會趨近於黃金比例1.61832...。

有一天呢,某個瘋狂的蘿糕同學想要製作一種可以毀滅疑中的武器,於是他在疑中放了許多定時炸彈,只要沒有在時間內解出密碼,炸彈就會爆炸!碰巧那個人非常喜歡費波納契數列,於是他將密碼設定為費波納契數列的前$N$項和。

怎麼辦,時間所剩不多,趕快寫一個程式來挽救疑中吧!

假設$f(n)$代表費波納契數列第n項,則$f(n)=f(n-1)+f(n-2),n>2$

本題中定義$f(1)=1,f(2)=1,f(3)=2$

\section*{Input}

輸入共一正整數$N$


\section*{Output}

請輸出費波納契數列的前$N$項和,亦即$\displaystyle\sum^N_{k=1}f(k)$

\section*{Sample 1}
\begin{longtable}[!h]{|p{0.5\textwidth}|p{0.5\textwidth}|}
\hline
\textbf {Input}	& \textbf {Output} \\
\hline
\parbox[t]{0.5\textwidth} % sample 1
{ \tt
5
} &
\parbox[t]{0.5\textwidth}
{ \tt
%output
12

} \\
\hline
\end{longtable}


\section*{Subtasks}

在一個子任務的「測試資料範圍」的敘述中,如果存在沒有提到範圍的變數,則此變數的範圍為 Input 所描述的範圍。

\begin{center}
 \begin{tabular}{||c c c||} 
 \hline
 子任務編號 & 子任務配分 & 測試資料範圍 \\  
 \hline
 \hline
 1 & 20\% & $N<11$ \\ 
 \hline
 %O(n^4)
 2 & 30\% & $N<40$\\
 \hline 
 % O(n^3)
 3 & 50\% & $N<90$ \\
 \hline

\end{tabular}
\end{center}

\end{document}
