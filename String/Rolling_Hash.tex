\documentclass[12pt,a4paper]{article}
\usepackage[margin=2cm]{geometry}
\usepackage{xeCJK}
\usepackage{fontspec}
\setCJKmainfont{Noto Serif CJK TC}[Script=CJK]
\setCJKmonofont{Noto Sans Mono CJK TC}[Script=CJK]
\usepackage{amsmath,amssymb}
\usepackage{graphicx}
\usepackage{fancyhdr}
\setlength{\headheight}{14.5pt}
\addtolength{\topmargin}{-2.5pt}
\usepackage{hyperref}
\usepackage{xcolor}
\usepackage{listings}
\usepackage{enumitem}
\usepackage{titlesec}
\usepackage{caption}
\usepackage{indentfirst}
\usepackage{tikz}
\setlength{\parindent}{2e
m}
\pagestyle{fancy}
\fancyhf{}
\cfoot{\thepage}
\linespread{1.3}
\fancyhead[L]{\nouppercase{\leftmark}}

\lstset{
    basicstyle=\ttfamily\footnotesize,  % 字型與大小
    keywordstyle=\color{blue},
    commentstyle=\color{gray},
    stringstyle=\color{orange},
    numbers=left,                       % 行號在左側
    numberstyle=\tiny\color{gray},
    stepnumber=1,                       % 每行都顯示行號
    numbersep=5pt,
    backgroundcolor=\color{white},
    frame=single,                       % 加上框線
    breaklines=true,                    % 長行自動換行
    tabsize=2,
    language=C++                     % 可以換成 \texttt{C++}, \texttt{Java}, etc.
}


\title{Rolling Hash}
\author{TAI, WEI-HSUAN}
\date{\today}

\begin{document}

\maketitle

\lhead{Rolling Hash}
\rhead{TAI, WEI-HSUAN}
\newpage
\tableofcontents
\newpage

\section*{Abstract}

\begin{abstract}
本講義旨在深入探討 Rolling Hash 字串搜尋演算法。
Rolling Hash演算法是一種基於Hash技術的字串搜尋方法,能夠在大規模文本中快速定位模式串的位置。
\end{abstract}

\section{引言}
字串搜尋是電腦科學中一個非常基礎且重要的問題:如何在一個長字串(文本,$T$)中找到一個短字串(模式串,$P$)的所有出現位置?
最直觀的方法是樸素(Naive)演算法,但其在某些情況下效率較低。KMP 演算法正是為了解決這個效率問題而提出。

\subsection{樸素字串搜尋演算法的缺陷}
考慮在文本 $T = \texttt{ABABCABAB}$ 中搜尋模式串 $P = \texttt{ABCABAB}$。
\begin{itemize}
    \item 樸素演算法從 $T$ 的開頭開始,逐一比較 $P$ 和 $T$ 的子字串。
    \item 如果在某個位置不匹配,則將 $P$ 向右移動一位,重新從 $P$ 的開頭與 $T$ 的當前位置進行比較。
\end{itemize}

例如:
\begin{itemize}
    \item $T$: \verb|A B A B C A B A B|
    \item $P$: \verb|A B C A B A B|
    \item 第一次匹配:\verb|T[0]| 與 \verb|P[0]| 匹配 (\verb|A=A|),\verb|T[1]| 與 \verb|P[1]| 匹配 (\verb|B=B|)。
    \item \verb|T[2]| (\verb|A|) 與 \verb|P[2]| (\verb|C|) 不匹配。此時,樸素演算法會將 $P$ 向右移動一位,從 \verb|T[1]| 重新開始比較。
\end{itemize}
這種「回溯」行為會導致大量的重複比較,尤其是在模式串中存在重複字元時,效率會非常低。其最差時間複雜度可達 $O(m \times n)$,其中 $n$ 是文本長度,$m$ 是模式串長度。

\section{Rolling Hash演算法}

\subsection{概述}

在前綴和的題目中,我們可以透過預處理來快速計算子字串的和。類似地,Rolling Hash 演算法利用雜湊函數來快速計算子字串的雜湊值,從而實現高效的字串匹配。

Rolling Hash 的核心思想為,使用雜湊 (Hash) 的方式將一個字串紀錄為一個整數,透過比較整數來判斷字串是否相等。
這樣,我們可以在 $O(1)$ 的時間內比較兩個字串是否相等。
由於 Hash 函數的特性,我們可以保證不同的字母在不同位置的組合會產生不同的雜湊值。如此一來,可以最大程度的保證不同字串雜湊值的唯一性。

\subsection{先備知識}

\subsubsection{費馬小定理}

已知$p$是一個質數,$a$是一個正整數且不是$p$的倍數,則有:
$$a^p\equiv a \mod p$$
這意味著 $a^p$ 除以 $p$ 的餘數等於 $a$。

\subsubsection{模逆元}
已知$p$是一個質數,$a$是一個正整數且不是$p$的倍數,則存在一個整數$b = a^{p-2} \mod p$,使得:
$$a \cdot b \equiv 1 \mod p$$
這個整數$b$稱為$a$在模$p$下的逆元,通常用符號$a^{-1}$表示。

\subsection{Hash 函數}

在 Rolling Hash 中,我們使用一個多項式雜湊函數來計算字串的雜湊值。對於字串 $S = s_0 s_1 \ldots s_{m-1}$,其雜湊值定義為:
$$H(S)=\displaystyle\sum_{i=0}^{|S|-1} s_i \cdot x^i \mod p$$

其中 $S$ 是字串的長度, $s_i$ 是字串中第 $i$ 個字母轉換的編號, $x$ 是一個正整數, $p$ 是一個質數。使用這種方式計算 Hash ,我們可以確保每個字母在不同位置的組合會產生不同的雜湊值,
進而確保不同字串雜湊值的唯一性。

讀者看到這邊,可能會感到疑惑:為什麼透過計算字串的雜湊值可以快速查找子字串?

透過 Hash 的方式,我們可以把一個字串映射到唯一的一個整數,因此,若我們能快速查找某個字串當中子字串的雜湊值,是否就能快速查找子字串是否存在於某個字串當中呢?

\subsection{Rolling Hash 前綴和}

為了能在 $O(1)$ 的時間內計算任意子字串的雜湊值,我們可以使用前綴和的思想。具體來說,我們可以預先計算出文本 $T$ 的每個前綴的湊值,
然後利用這些前綴雜湊值來快速計算任意子字串的雜湊值。

假設我們有字串 $T$, $T[a, b]$代表字串 $T$ 中從位置 $a$ 到位置 $b$ 的子字串,為了能在$O(1)$的時間內找出 $H(T[a, b])$,
我們可以先進行一個 $O(T)$ 的預處理,定義 $S[N] = H(T[0, N])$,則透過以下的步驟:

$$S[0] = T[0]$$
$$S[i] = (S[i-1] + T[i]\cdot x^i) \mod p$$

可以在 $O(T)$ 的時間內計算出 $S$ 完成預處理。當我們要查找 $H(T[a, b])$ 時,只需要透過以下的公式:

$$S[b] - S[a-1] \mod p = \displaystyle\sum_{i=0}^{b-1}T_ix^i - \displaystyle\sum_{i=0}^{a-1}T_ix^i \mod p = \displaystyle\sum_{i=a}^{b-1}T_ix^i \mod p = x^{a}\cdot H(T[a, b])$$

即可在 $O(1)$ 的時間內計算出 $H(T[a, b])$。

方才的公式得到的 $H(T[a, b])$ 並不乾淨,還多了一個 $x^{a}$ 的因子,因此我們需要計算 $x^{a}$ 的模逆元 $x^{-a} \bmod p = (x^a)^{p-2} \bmod p$,
然後將其乘上 $S[b] - S[a-1]$,即可得到乾淨的 $H(T[a, b])$。

最後,可以得到:

$$H(T[a, b]) = x^{-a} \cdot (S[b] - S[a-1]) \mod p$$


\subsection{Rolling Hash 碰撞分析}

雖然 Rolling Hash 可以在 $O(1)$ 的時間內計算任意子字串的雜湊值,但仍然存在碰撞的可能性,即不同的字串可能會有相同的雜湊值。
這是由於 Hash 函數的特性所導致的。為了減少碰撞的機率,我們可以選擇一個較大的質數 $p$ 和一個合適的基數 $x$。

碰撞發生的概率不大,但仍然存在。為了進一步降低碰撞的概率,我們可以使用雙重雜湊(Double Hashing)技術,
即對同一個字串使用兩個不同的雜湊函數,只有當兩個雜湊值都相等時,才認為兩個字串相等。


\subsection{C++ 實作}

給定一個字串 $T$ 和一個模式串 $P$,求出 $P$ 在 $T$ 中出現的次數。

\lstset{caption={Rolling Hash 演算法的 C++ 實作}}
\begin{lstlisting}[language=C++]
vector<int> prepross(string s){
    vector<int> pre(s.size()+5);

    pre[0]=(int)(s[0]-'a');
    int now_x=1;
    for(int i=1;i<s.size();++i){
        now_x=now_x*x%p;
        pre[i]=(pre[i-1]+(int)(s[i]-'a')*now_x)%p;
    }
    return pre;
}

int query(vector<int> pre, int a, int b){
    if(a==0)return pre[b];
    int inverse=fpow(fpow(x, a), p-2);
    return ((pre[b]-pre[a-1]+p)*inverse)%p;
}

int rolling_hash(vector<int> pre, string s, string t){
    int ss=s.size(), ts=t.size(), cnt=0;
    if(ts>ss)return 0;

    int hash_t=0, now_x=1;
    for(int i=0;i<t.size();++i){
        hash_t=(hash_t+(int)(t[i]-'a')*now_x)%p;
        now_x=(now_x*x)%p;
    }

    for(int i=0;i+ts-1<ss;++i){
        int j=i+ts-1;
        cnt+=(query(pre, i, j)==hash_t);
    }

    return cnt;
}
\end{lstlisting}

\section{總結}
Rolling Hash 演算法是一種高效率的演算法,能夠在 $O(n)$ 的時間內解決字串搜尋問題。
透過前綴和的思想,我們可以在 $O(1)$ 的時間內計算任意子字串的雜湊值,從而實現快速的字串匹配。

Hash的部份並沒有硬性規定,只要能夠保證不同字串的雜湊值唯一即可。為了避免產生碰撞,我們可以選擇一個較大的質數 $p$ 和一個合適的基數 $x$
,或是使用雙重雜湊(Double Hashing)技術。


\section{總結}
KMP 演算法利用模式串自身的重複特性,避免了樸素演算法中不必要的回溯,從而將字串搜尋的時間複雜度優化到線性的 $O(m+n)$。其核心在於構建一個「部分匹配表」(LPS 陣列),該表指導了模式串在不匹配時如何高效地跳躍。掌握 KMP 演算法不僅能解決字串匹配問題,其背後的動態規劃思想和對問題結構的利用,對於理解其他演算法也大有裨益。
