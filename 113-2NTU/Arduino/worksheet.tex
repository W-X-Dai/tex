\documentclass[12pt,a4paper]{article}
\usepackage[margin=2cm]{geometry}
\usepackage{xeCJK}
\usepackage{fontspec}
\setCJKmainfont{Noto Serif CJK TC}[Script=CJK]
\usepackage{amsmath,amssymb}
\usepackage{graphicx}
\usepackage{fancyhdr}
\setlength{\headheight}{14.5pt}
\addtolength{\topmargin}{-2.5pt}
\usepackage{hyperref}
\usepackage{listings}
\usepackage{enumitem}
\usepackage{titlesec}
\usepackage{caption}
\usepackage{indentfirst}
\usepackage{array}
\setlength{\parindent}{2em}
\pagestyle{fancy}
\fancyhf{}
\cfoot{\thepage}
\linespread{1.3}

\usepackage{xcolor}
\lstset{
    basicstyle=\ttfamily\footnotesize,  % 字型與大小
    keywordstyle=\color{blue},
    commentstyle=\color{gray},
    stringstyle=\color{orange},
    numbers=left,                       % 行號在左側
    numberstyle=\tiny\color{gray},
    stepnumber=1,                       % 每行都顯示行號
    numbersep=5pt,
    backgroundcolor=\color{white},
    frame=single,                       % 加上框線
    breaklines=true,                    % 長行自動換行
    tabsize=2,
    language=C++                     % 可以換成 \texttt{C++}, \texttt{Java}, etc.
}

\usepackage{titlesec}
\titleformat{\section}[block]{\normalfont\Large\bfseries}{}{0pt}{}
\makeatletter
\renewcommand{\thesubsection}{\@arabic\c@subsection}
\makeatother

\begin{document}

\lhead{電子學實驗手冊}
\rhead{2025/07/05}


\section{電子學實驗手冊}

\subsection{\texttt{LED}燈}

\subsubsection{點亮燈泡}
請使用你手邊的工具將一個燈泡點亮,你可以參考以下的程式碼來完成這個任務。
\begin{lstlisting}
void setup() {
    pinMode(3, OUTPUT);
    digitalWrite(3, HIGH);
}
\end{lstlisting}

\subsubsection{讓燈泡閃爍吧}
在\texttt{Arduino}中,\texttt{delay(t)}函數可以讓程式暫停\texttt{t}毫秒,
巧妙利用這個函數以及\texttt{digitalWrite()}函數,可以讓燈泡閃爍起來。

現在,請你利用剛剛1.1的電路,讓燈泡閃爍吧!

\subsection{蜂鳴器}

\subsubsection{讓蜂鳴器叫一下}
請修改你在1.2的作品,讓蜂鳴器在燈泡閃爍的同時發出聲音。

\subsubsection{蜂鳴器的音調}
蜂鳴器的功能並不只是發出聲音而已,你也可以配合\texttt{tone()}函數來改變音調。

\texttt{Arduino}提供了一個函數\texttt{tone(pin, frequency)},
利用控制腳位輸出的電壓頻率使蜂鳴器發出不同的音調(你可以用\texttt{LED}燈來試試)。

舉例來說:
\begin{lstlisting}
tone(9, 440); //在第9腳位輸出440Hz(也就是音樂中的「A4」)
\end{lstlisting}

此外,你也可以搭配\texttt{delay()}函數來控制音調的持續時間,
要記得這個函式控制的時間單位是毫秒(ms),也就是千分之一秒。

現在,請使用手邊的蜂鳴器演奏一段旋律吧!

以下是不同音調的對照表,你可以參考這個表格來設計你的旋律:
\begin{table}[h!]
\centering
\begin{tabular}{|c|c|c|c|c|c|c|c|c|c|c|c|}
\hline
Note & C4 & D4 & E4 & F4 & G4 & A4 & B4 & C5 & D5 & E5 & F5 \\ \hline
Freq (Hz) & 261 & 294 & 329 & 349 & 392 & 440 & 493 & 523 & 587 & 659 & 698 \\ \hline
\end{tabular}
\caption{音調對照表}
\label{tab:freq_table}
\end{table}

\subsection{觸控開關}
\subsubsection{你真的有按到嘛?}
請設計一個電路及程式碼,按下開關時讓燈泡亮起,放開時燈泡熄滅。

\subsubsection{超酷炫的手電筒}
Josh想要設計一個超酷炫的手電筒,他的要求有:
\begin{itemize}
    \item 沒開燈的時候點擊開關會開燈,否則關燈
    \item 點擊開關的同時蜂鳴器會叫一下
    \item 如果你點擊開關超過三秒的話燈泡會閃爍
\end{itemize}

但是他正在被一大堆的死線追殺,沒時間設計這個超級酷炫的手電筒,請幫幫他吧!

\subsubsection{求救信號}
正在讀期末的可悲大學生感覺自己快要不行了,請設計一個電路及程式碼,
讓他按下開關就可以控制\texttt{LED}燈泡閃爍,並且發出\texttt{SOS}的求救信號,
發出求救信號之後請控制蜂鳴器發出救護車的音調。
救救可悲的大學牲吧!

\texttt{SOS}的摩斯密碼是\texttt{... --- ...},也就是三個點、三個劃、三個點;
救護車的音調為715Hz 和956Hz交替發出。

下面提供其他摩斯密碼的對照表,你也可以設計你自己的求救信號:
\begin{table}[h!]
\centering
\begin{tabular}{|c|c|c|c|c|c|c|c|c|c|c|}
\hline
Letter & A   & B    & C    & D   & E   & F   & G    & H    & I    & J    \\ \hline
Code  & \texttt{.-}  & \texttt{-...} & \texttt{-.-.} & \texttt{-..} & \texttt{.}   & \texttt{..-.} & \texttt{--.}  & \texttt{....} & \texttt{..}   & \texttt{.---} \\ \hline
Letter & K   & L    & M    & N   & O   & P   & Q    & R    & S    & T    \\ \hline
Code  & \texttt{-.-} & \texttt{.-..} & \texttt{--}   & \texttt{-.}  & \texttt{---} & \texttt{.--.} & \texttt{--.-} & \texttt{.-.}  & \texttt{...}  & \texttt{-}    \\ \hline
Letter & U   & V    & W    & X   & Y   & Z   & 0    & 1    & 2    & 3    \\ \hline
Code  & \texttt{..-} & \texttt{...-} & \texttt{.--}  & \texttt{-..-} & \texttt{-.--} & \texttt{--..} & \texttt{-----} & \texttt{.----} & \texttt{..---} & \texttt{...--} \\ \hline
Letter & 4   & 5    & 6    & 7   & 8   & 9   &      &      &      &      \\ \hline
Code  & \texttt{....-} & \texttt{.....} & \texttt{-....} & \texttt{--...} & \texttt{---..} & \texttt{----.} &      &      &      &      \\ \hline
\end{tabular}
\caption{摩斯密碼對照表}
\label{tab:morse_table}
\end{table}


\newpage
\subsection{挑戰題(Bonus)}

利用手邊的工具製作一個「摩斯密碼打字機」,利用按壓觸控開關時間的長短來決定是點(.)還是劃(-)
,並且能夠將輸入的摩斯密碼轉換成對應的字母顯示在序列埠監視器(Serial Monitor)上。

假設這個觸控開關的正極、負極已經連接在開發板的\texttt{5V}和\texttt{GND},
並且\texttt{SIG}接在開發板的腳位7;腳位8連接了一個\texttt{LED}燈泡,如果開關被按壓,燈泡就會亮。
請完成以下程式碼的填空:

\begin{lstlisting}
#define BUTTON_PIN 7
#define THRESHOLD 200 //定義點的時間,小於此值為點,大於為劃

String morseCode = "",code="";
unsigned long buttonPressTime = 0;
bool isPressed = false;

//摩斯密碼對照表
const char* morseTable[][2] = {
    {".-", "A"}, {"-...", "B"}, {"-.-.", "C"}, {"-..", "D"}, {".", "E"},
    {"..-.", "F"}, {"--.", "G"}, {"....", "H"}, {"..", "I"}, {".---", "J"},
    ...(以下略)
};

void setup() {
    pinMode(___[1]___, INPUT);
    pinMode(8, OUTPUT);
    Serial.begin(9600); //初始化通信埠
}

inline char transCode(String morseCode) {
    //搜尋整個對照表
    for (int i = 0; i < 36; ++i) {
        //輸入的內容和對照表上的某個值相同
        if (String(morseTable[i][0]) == morseCode) {
            return morseTable[i][1][0]; //回傳對應的字母
        }
    }
    return '?'; //如果輸入的東西不存在於對照表,回傳問號
}

void loop() {
    int buttonState = digitalRead(___[2]___);
    if(___[3]___)digitalWrite(8,HIGH);
    else digitalWrite(8,LOW);

    //原本沒壓下去,紀錄壓下去那一刻的時間戳記
    if(buttonState and !isPressed){
        buttonPressTime=millis();  //millis()會回傳當前的時間(單位:ms)
        isPressed=1;
    }
    
    //原本的狀態是壓下去的,此時放開,計算按壓時間持續了多久
    if(!buttonState and isPressed){
        unsigned long pressDuration=millis()-buttonPressTime;
        //持續時間小於THRESHOLD就紀錄為點,否則為劃
        if(___[4]___<THRESHOLD)Serial.print("."), morseCode+=".";
        else Serial.print("-"), morseCode+="-";
        isPressed=0;
    }
    
    //停止輸入一秒後,把曾經輸入的點、劃轉換成字母
    if(!isPressed and buttonPressTime and millis()-buttonPressTime>1000){
        char new_letter=transCode(___[5]___);//把得到的點-劃字串轉換成字母
        code+=new_letter; //把新轉譯出來的字母加到結果
        buttonPressTime=0; //重製計時器
        morseCode = ""; //清理暫存
        Serial.print("目前輸入:");
        Serial.println(___[6]___); //顯示結果
    }
}
\end{lstlisting}

答題表:
\begin{table}[h!]
\centering
\begin{tabular}{|c|p{7cm}|c|p{7cm}|}
\hline
1 &  & 4 & \\ \hline
2 &  & 5 & \\ \hline
3 &  & 6 & \\ \hline
\end{tabular}
\end{table}

\section{分數計算表}

分數計算:
每完成一個實驗可以得10分,總分70分;挑戰題每答對一個得5分,如果全部答對則額外獲得10分,總分40分。
因此,在這個實驗你最多可以得到110分,除了挑戰題以外,其他的實驗必須要在課程中完成。

\begin{table}[h!]
\centering
\begin{tabular}{|c|c|>{\centering\arraybackslash}p{3cm}|}
\hline
項目 & 完成了什麼 & 總分 \\ \hline
LED燈 & 1-1,1-2  &  \\ \hline
蜂鳴器 &  2-1,2-2 &  \\ \hline
觸控開關 & 3-1,3-2,3-3 &  \\ \hline
\end{tabular}
\end{table}

\end{document}