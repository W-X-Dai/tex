\documentclass[12pt,a4paper]{article}
\usepackage[margin=2cm]{geometry}
\usepackage{xeCJK}
\usepackage{fontspec}
\setCJKmainfont{Noto Serif CJK TC}[Script=CJK]
\usepackage{amsmath,amssymb}
\usepackage{graphicx}
\usepackage{fancyhdr}
\setlength{\headheight}{14.5pt}
\addtolength{\topmargin}{-2.5pt}
\usepackage{hyperref}
\usepackage{listings}
\usepackage{enumitem}
\usepackage{titlesec}
\usepackage{caption}
\usepackage{indentfirst}
\setlength{\parindent}{2em}
\pagestyle{fancy}
\fancyhf{}
\cfoot{\thepage}
\linespread{1.3}

\begin{document}

\lhead{醫工實驗期末心得}
\rhead{B12508026戴偉璿}

我認為這堂課中,學到最重要的東西是「問問題」。

有時候,「問問題」會比死讀書還要來的重要許多。
即使配備了數十萬美金的高級顯卡,大型語言模型仍然要透過自注意力機制集中注意力於幾個重點,更遑論是肉體凡胎的人腦-我們無法一次性注意到所有的細節。
而「問問題」,則是我們在學習的過程中,集中注意力的方式。當注意力沒那麼多的時候,我們就要嘗試將所有注意力集中在關鍵點,而這個關鍵點就是自己無法理解的地方。
因此,在學習的過程中,我總是不停反問自己「為什麼」會這樣,集中火力在自己無法理解的地方,才是最有效率的方法。

然而,並不是所有時候都能夠輕而易舉地找到問題的關鍵點。有時在看一段文字的時候會下意識把某些事情當作理所當然,
比如說看一個數學公式的推導,或許當下看懂了,但把書闔上後,卻發現自己無法從零開始推導出整個過程,這時候就需要重新審視自己的理解,找出問題的根源。
而老師在課堂上提到的拆解的方法,或許是個很管用的工具。

「如果你回到古代,手邊有所有的材料,為什麼無法造出一個聽診器?」這個問題看似荒謬,但其實是個很好的思考練習。
我們使用工具使用的太過於順手,往往忽略了背後的原理和知識。就算有組裝的技術,有動手的能力,為什麼造出來的聽診器品質會差那麼多?
或許是聲音傳導的原理,或許是使用聽診器的方法不對,或許是材料的選擇不當,或許是組裝的技術不夠好。
這些問題都可以讓我們更深入地思考,嘗試去理解背後的本質。

醫學工程是一個跨領域的學科,涉及到醫學、工程學、物理學等多個領域的知識。
世界上沒有任何一種材料的複雜度比人體還要高,
因此在設計醫療器材的時候,難度又會再增加,除了考量到器材的功能,還要考量到人體的生理結構和生理反應。
更甚者,還要考量到「病人」這個變數,因為每個病人的生理結構和生理反應都不盡相同。
這就需要我們在設計醫療器材的時候,能夠靈活地運用各種知識,並且能夠針對不同的病人做出調整。
因此,學會拆解出問題的本質,並且能夠靈活地運用各種知識,是醫學工程領域中非常重要的能力。

在過往的幾十幾百年間,各種材料的特性已經被摸透了,甚至有軟體可以直接計算出建築物的強度和穩定性。
但人體的複雜度遠超過這些材料,人體的生理結構和生理反應仍然有很多未知的領域。
若要成為一名優秀的醫學工程師,除了要有扎實的基礎知識,靈活的思維和創新的能力也是必備的。

\end{document}