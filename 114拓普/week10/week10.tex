\documentclass[xcolor=dvipsnames]{beamer}

% ==== 主題 ====
\usetheme{metropolis}
\usefonttheme{professionalfonts}          % 不覆蓋你自訂的字型

% ==== 字型 ====
\usepackage{fontspec}
\usepackage{xeCJK}
\renewcommand{\familydefault}{\rmdefault} % 使用 serif 字體(重點)

% 西文字型:Times New Roman 的開源替代品
\setmainfont{TeX Gyre Termes}[
  Ligatures=TeX,
  BoldFont={* Bold},
  ItalicFont={* Italic}
]

% 中文字型(可改為思源宋體、標楷體等)
\setCJKmainfont{Noto Serif CJK TC}
\setCJKsansfont{Noto Sans CJK TC} % 有需要再用
\setCJKmonofont{Noto Sans Mono CJK TC}

% ==== 數學字型(與正文字體一致)====
\usepackage{unicode-math}
\setmathfont{TeX Gyre Termes Math}

% ==== 套件 ====
\usepackage{amsmath, amssymb}
\usepackage{graphicx}
\usepackage{hyperref}
\usepackage{minted}
\usepackage{fvextra}
\usepackage{xcolor}
\usepackage{booktabs}

% ==== 顏色設定(可選)====
\definecolor{MyBlue}{RGB}{3, 55, 105}
\setbeamercolor{structure}{fg=MyBlue}
\setbeamercolor{block title}{bg=MyBlue,fg=white}
\setbeamercolor{block body}{bg=blue!5}
\setbeamertemplate{section in toc}{%
  \inserttocsectionnumber.~\inserttocsection\par
}

\setminted{
    linenos,                % 行號
    frame=lines,            % 上下框線
    framesep=5pt,           % 程式碼與邊框距離
    numbersep=8pt,          % 行號與程式碼距離
    fontsize=\scriptsize,   % 字體大小
    breaklines,             % 自動換行
    tabsize=4,              % tab 寬度
    rulecolor=\color{black},% 框線顏色
    xleftmargin=1.5em       % 左側縮排
}

\title{APCS Practice}
\author{Tai, Wei Hsuan}
\date{week 10}
\AtBeginSubsection{
  \begin{frame}{Outline}
    \tableofcontents[currentsection, currentsubsection]
  \end{frame}
}

\begin{document}
\begin{frame}
    \titlepage
\end{frame}

\begin{frame}
    \frametitle{Announcement}
    Recording or photographing the slides or any content displayed on the screen is permitted for personal use only. \\[6pt]
    Redistribution, modification, or any commercial use of the materials is strictly prohibited. \\[12pt]
    \textit{© 2025 [Tai, Wei Hsuan]. All rights reserved. For personal use only.}
\end{frame}

    % course material qrcode
    \begin{frame}
        \frametitle{\href{https://drive.google.com/drive/folders/14Tkn-rddw0k1obeOxkWi00S43M0e9wlW?usp=sharing}{Course Materials}}
        \begin{figure}
            \centering
            \includegraphics[width=0.6\textwidth]{src/qrcode.png}
        \end{figure}
    \end{frame}

    \begin{frame}
        \frametitle{Outline}
        \tableofcontents
    \end{frame}
    \section{Intro to APCS}
    \begin{frame}
        \frametitle{What is APCS}
        \begin{itemize}
            \item Advanced Placement Computer Science (APCS)
            \item Help you apply to universities with a lower grade 
            \item Coding exam and multiple choice exam
        \end{itemize}
    \end{frame}
    \begin{frame}
        \frametitle{Coding Exam}
        \begin{itemize}
            \item Seperated to 4 levels(初級、中級、中高級、高級)
            \item Each level has different problems.
            \item You have to finish 3 problems in 2 hours.
            \item \textbf{OFFLINE JUDGE}
            \item My target is to help you pass the \textbf{初級} and \textbf{中級} level in G7.
        \end{itemize}
    \end{frame}
    \begin{frame}
        \begin{figure}
            \centering
            \includegraphics[width=\textwidth]{src/apcs_levels.png}
            \caption{APCS Levels}
        \end{figure}
    \end{frame}
    \begin{frame}
        \begin{figure}
            \centering
            \includegraphics[width=\textwidth]{src/basic_level.png}
            \caption{初級測驗範圍}
        \end{figure}
    \end{frame}
    \begin{frame}
        \begin{figure}
            \centering
            \includegraphics[width=\textwidth]{src/mid_level.png}
            \caption{中級測驗範圍}
        \end{figure}
    \end{frame}
    
    \section{Sample Problems}
    \begin{frame}
        \frametitle{g275. 七言對聯}
        Just determine by the requirements.
    \end{frame}
    \begin{frame}
        \frametitle{o711. 裝飲料}
        This is a little complex, but the problem can be seperated into 4 conditions, set the bottom one has L1 capacity, the second one has L2 capacity.
        \begin{itemize}
            \item L1 $\rightarrow$ L1
            \item L1 $\rightarrow$ L2
            \item L1 $\rightarrow$ capacity of the cup
            \item L2 $\rightarrow$ L2
            \item L2 $\rightarrow$ capacity of the cup
        \end{itemize}
    \end{frame}
    \begin{frame}
        \frametitle{m931. 遊戲選角}

    \end{frame}
\end{document}