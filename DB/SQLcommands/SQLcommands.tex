\documentclass[12pt,a4paper]{article}
\usepackage[margin=2cm]{geometry}
\usepackage{xeCJK}
\usepackage{fontspec}
\setCJKmainfont{Noto Serif CJK TC}[Script=CJK]
\usepackage{amsmath,amssymb}
\usepackage{graphicx}
\usepackage{fancyhdr}
\setlength{\headheight}{14.5pt}
\addtolength{\topmargin}{-2.5pt}
\usepackage{hyperref}
\usepackage{listings}
\usepackage{enumitem}
\usepackage{titlesec}
\usepackage{caption}
\usepackage{indentfirst}
\usepackage{float}
\usepackage{forest}
\setlength{\parindent}{2em}
\pagestyle{fancy}
\fancyhf{}
\cfoot{\thepage}
\linespread{1.3}

\usepackage{multirow}
\usepackage{booktabs}
\usepackage{tikz}
\usetikzlibrary{shapes,positioning,calc}
\colorlet{lightgray}{gray!20}

\usepackage{minted}
\setminted{
    linenos,
    frame=lines,
    framesep=5pt,
    numbersep=8pt,
    fontsize=\scriptsize,
    breaklines,
    tabsize=4,
    rulecolor=\color{black},
    xleftmargin=1.5em
}

\begin{document}

\section*{常見 SQL 指令總表}

以下整理自簡報內容:contentReference[oaicite:0]{index=0},分為資料定義、資料操作與進階查詢三大類。

\subsection*{1. 資料定義 (DDL)}
\begin{itemize}
  \item \textbf{CREATE}:建立資料庫物件
  \begin{minted}{sql}
CREATE TABLE EMPLOYEE (
  id VARCHAR(15) PRIMARY KEY,
  name VARCHAR(50) NOT NULL,
  birthday DATE NOT NULL
);
  \end{minted}

  \item \textbf{ALTER}:修改資料表
  \begin{minted}{sql}
ALTER TABLE EMPLOYEE ADD job VARCHAR(20);
ALTER TABLE EMPLOYEE DROP COLUMN birthday CASCADE;
  \end{minted}

  \item \textbf{DROP}:刪除資料表或 schema
  \begin{minted}{sql}
DROP TABLE EMPLOYEE CASCADE;
DROP SCHEMA RETAIL CASCADE;
  \end{minted}
\end{itemize}

\subsection*{2. 基本查詢 (SELECT)}
\begin{itemize}
  \item \textbf{基本語法}
  \begin{minted}{sql}
SELECT <屬性清單>
FROM <資料表>
WHERE <條件>;
  \end{minted}

  \item \textbf{條件與運算子}
  \begin{minted}{sql}
-- 等於 / 不等於
SELECT * FROM EMPLOYEE WHERE gender = 'M';
SELECT * FROM EMPLOYEE WHERE id <> 'A123';

-- LIKE 與萬用字元
SELECT name FROM EMPLOYEE WHERE name LIKE 'Chi%';

-- NULL 判斷
SELECT name FROM EMPLOYEE WHERE supervisor_id IS NULL;
  \end{minted}

  \item \textbf{集合運算}
  \begin{minted}{sql}
SELECT salary FROM EMPLOYEE WHERE store_id = 1
UNION
SELECT salary FROM EMPLOYEE WHERE store_id = 2;
  \end{minted}

  \item \textbf{JOIN}
  \begin{minted}{sql}
-- 內部連接
SELECT e.name, s.postal_code
FROM EMPLOYEE e
JOIN STORE s ON e.store_id = s.id;

-- 左外連接
SELECT e.name, s.mgr_id
FROM EMPLOYEE e
LEFT JOIN STORE s ON e.id = s.mgr_id;
  \end{minted}

  \item \textbf{排序與聚合}
  \begin{minted}{sql}
SELECT store_id, COUNT(*), AVG(monthly_salary)
FROM EMPLOYEE
GROUP BY store_id
HAVING COUNT(*) > 2
ORDER BY AVG(monthly_salary) DESC;
  \end{minted}
\end{itemize}

\subsection*{3. 資料操作 (DML)}
\begin{itemize}
  \item \textbf{INSERT}
  \begin{minted}{sql}
INSERT INTO EMPLOYEE(id, name, gender, birthday, monthly_salary)
VALUES('A123456789', 'Xiao-Ming Wang', 'M', '1999-01-01', 120000);
  \end{minted}

  \item \textbf{DELETE}
  \begin{minted}{sql}
DELETE FROM EMPLOYEE WHERE name = 'Chih-Yuan Lee';
  \end{minted}

  \item \textbf{UPDATE}
  \begin{minted}{sql}
UPDATE EMPLOYEE
SET monthly_salary = monthly_salary * 1.1
WHERE store_id = 1;
  \end{minted}
\end{itemize}

\subsection*{4. 進階查詢}
\begin{itemize}
  \item \textbf{子查詢與集合比較}
  \begin{minted}{sql}
SELECT name
FROM EMPLOYEE
WHERE monthly_salary > ALL (
  SELECT monthly_salary FROM EMPLOYEE WHERE store_id = 1
);
  \end{minted}

  \item \textbf{LIMIT \& OFFSET}
  \begin{minted}{sql}
SELECT name FROM EMPLOYEE
ORDER BY birthday ASC
LIMIT 3 OFFSET 1;
  \end{minted}

  \item \textbf{CASE 條件判斷}
  \begin{minted}{sql}
SELECT id, name,
CASE SUBSTRING(id,1,1)
  WHEN 'A' THEN 'Taipei'
  WHEN 'S' THEN 'Kaohsiung'
  ELSE 'Others'
END AS region
FROM EMPLOYEE;
  \end{minted}

  \item \textbf{CTE (Common Table Expression)}
  \begin{minted}{sql}
WITH MaxSalary AS (
  SELECT MAX(monthly_salary) AS max_sal
  FROM EMPLOYEE WHERE store_id = 2
)
SELECT name FROM EMPLOYEE
WHERE monthly_salary > (SELECT max_sal FROM MaxSalary);
  \end{minted}

  \item \textbf{VIEW}
  \begin{minted}{sql}
CREATE VIEW STORE_SALES_VIEW AS
SELECT s.store_id, sd.product_id, sd.qty
FROM STORE s
JOIN SALES_DETAIL sd ON s.id = sd.store_id;
  \end{minted}
\end{itemize}

\end{document}
