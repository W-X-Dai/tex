\documentclass[12pt,a4paper]{article}
\usepackage[margin=2cm]{geometry}
\usepackage{xeCJK}
\usepackage{fontspec}
\setCJKmainfont{Noto Serif CJK TC}[Script=CJK]
\usepackage{amsmath,amssymb}
\usepackage{graphicx}
\usepackage{fancyhdr}
\setlength{\headheight}{14.5pt}
\addtolength{\topmargin}{-2.5pt}
\usepackage{hyperref}
\usepackage{listings}
\usepackage{enumitem}
\usepackage{titlesec}
\usepackage{caption}
\usepackage{indentfirst}
\usepackage{booktabs}
\usepackage{longtable}
\usepackage{multirow}
\usepackage{array}
\usepackage{tabularx}
\usepackage{float}
\usepackage{minted}
\setlength{\parindent}{2em}
\pagestyle{fancy}
\fancyhf{}
\cfoot{\thepage}
\linespread{1.3}
\setminted{
    linenos,                % 行號
    frame=lines,            % 上下框線
    framesep=5pt,           % 程式碼與邊框距離
    numbersep=8pt,          % 行號與程式碼距離
    fontsize=\scriptsize,   % 字體大小
    breaklines,             % 自動換行
    tabsize=4,              % tab 寬度
    rulecolor=\color{black},% 框線顏色
    xleftmargin=1.5em       % 左側縮排
}

\title{Fundamentals of Biomedical Image Processing HW 3}
\author{B12508026戴偉璿}
\date{\today}

\begin{document}

\maketitle

\lhead{lhead}
\rhead{B12508026戴偉璿}

\section{Theoretical Questions}

To detect a 1-pixel break in a binary line, we can use a directional filter such as [1 -2 1]. When applied to a continuous line (1 1 1), the filter output is zero; when applied to the edge of the break (1 1 0), the output is negative (-1); when applied to a break (1 0 1), the output becomes positive (2). Therefore, pixels with nonzero responses indicate line breaks.

Here are the filters that can detect the 1-pixel break in different directions:

\begin{itemize}
    \item Vertical:
        $$A = \begin{bmatrix}
            0 & 1 & 0 \\
            0 & -2 & 0 \\
            0 & 1 & 0
            \end{bmatrix}
        $$
    \item Horizontal:
        $$A = \begin{bmatrix}
            0 & 0 & 0 \\
            1 & -2 & 1 \\
            0 & 0 & 0
            \end{bmatrix}
        $$
    \item +45 degree:
        $$A = \begin{bmatrix}
            0 & 0 & 1 \\
            0 & -2 & 0 \\
            1 & 0 & 0
            \end{bmatrix}
        $$
    \item -45 degree:
        $$A = \begin{bmatrix}
            1 & 0 & 0 \\
            0 & -2 & 0 \\
            0 & 0 & 1
            \end{bmatrix}
        $$
\end{itemize}

\section{Programming Exercises}




\end{document}