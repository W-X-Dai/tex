\documentclass[12pt,a4paper]{article}
\usepackage[margin=2cm]{geometry}
\usepackage{xeCJK}
\usepackage{fontspec}
\setCJKmainfont{Noto Serif CJK TC}[Script=CJK]
\usepackage{amsmath,amssymb}
\usepackage{graphicx}
\usepackage{fancyhdr}
\setlength{\headheight}{14.5pt}
\addtolength{\topmargin}{-2.5pt}
\usepackage{hyperref}
\usepackage{listings}
\usepackage{enumitem}
\usepackage{titlesec}
\usepackage{caption}
\usepackage{indentfirst}
\usepackage{booktabs}
\usepackage{longtable}
\usepackage{multirow}
\usepackage{array}
\usepackage{tabularx}
\usepackage{float}
\usepackage{minted}
\setlength{\parindent}{2em}
\pagestyle{fancy}
\fancyhf{}
\cfoot{\thepage}
\linespread{1.3}
\setminted{
    linenos,                % 行號
    frame=lines,            % 上下框線
    framesep=5pt,           % 程式碼與邊框距離
    numbersep=8pt,          % 行號與程式碼距離
    fontsize=\scriptsize,   % 字體大小
    breaklines,             % 自動換行
    tabsize=4,              % tab 寬度
    rulecolor=\color{black},% 框線顏色
    xleftmargin=1.5em       % 左側縮排
}

\title{直線與圓}
\author{B12508026戴偉璿}
\date{\today}

\begin{document}

\lhead{直線與圓}
\rhead{\today}

\begin{enumerate}
    \item 兩直線$L_1: (m+2)x+(m+3)y=10, L_2: 6x+(2m-1)y=5$
    \begin{enumerate}
        \item 計算$m$為何值時,$L_1$與$L_2$平行。\vspace{2cm}
        \item 計算$m$為何值時,$L_1$與$L_2$垂直。\vspace{2cm}
    \end{enumerate}
    \item 若$\triangle ABC$是個直角三角形且$A(3, -2), B(-1, a), C(2, 1)$,求$a$的值。\vspace{2cm}
    \item 已知$L_1: x-2y+3=0, L_2: 2x+3y=0, L_3: ax-y-1=0$,若這三條直線不能圍成三角形,求$a$的值。\vspace{2cm}
    \item 設$A(-2, -1), B(3, 2)$,直線$L: x-y+k=0$與$\overline{AB}$相交,求$k$的範圍。\vspace{2cm}
    \item 設$A(2, 1), B(-1, 3)$,直線$L: y=mx+6$與$\overline{AB}$相交,求$m$的範圍。\vspace{2cm}
    \item 已知$L_1: 2x+y-2=0, L_2: y=0, L_3: x-y+1=0$共同圍出了三角形$\triangle ABC$
    \begin{enumerate}
        \item 求$\triangle ABC$的面積。\vspace{2cm}
        \item 若直線$L: y=mx+\tfrac{3}{2}$與這個三角形相交,求$m$的範圍。\vspace{2cm}
    \end{enumerate}
\end{enumerate}
\end{document}