\documentclass[xcolor=dvipsnames]{beamer}

% ==== 主題 ====
\usetheme{metropolis}
\usefonttheme{professionalfonts}          % 不覆蓋你自訂的字型

% ==== 字型 ====
\usepackage{fontspec}
\usepackage{xeCJK}
\renewcommand{\familydefault}{\rmdefault} % 使用 serif 字體(重點)

% 西文字型:Times New Roman 的開源替代品
\setmainfont{TeX Gyre Termes}[
  Ligatures=TeX,
  BoldFont={* Bold},
  ItalicFont={* Italic}
]

% 中文字型(可改為思源宋體、標楷體等)
\setCJKmainfont{Noto Serif CJK TC}
\setCJKsansfont{Noto Sans CJK TC} % 有需要再用
\setCJKmonofont{Noto Sans Mono CJK TC}

% ==== 數學字型(與正文字體一致)====
\usepackage{unicode-math}
\setmathfont{TeX Gyre Termes Math}

% ==== 套件 ====
\usepackage{amsmath, amssymb}
\usepackage{graphicx}
\usepackage{hyperref}
\usepackage{minted}
\usepackage{fvextra}
\usepackage{xcolor}
\usepackage{booktabs}

% ==== 顏色設定(可選)====
\definecolor{MyBlue}{RGB}{3, 55, 105}
\setbeamercolor{structure}{fg=MyBlue}
\setbeamercolor{block title}{bg=MyBlue,fg=white}
\setbeamercolor{block body}{bg=blue!5}
\setbeamertemplate{section in toc}{%
  \inserttocsectionnumber.~\inserttocsection\par
}

\setminted{
    linenos,                % 行號
    frame=lines,            % 上下框線
    framesep=5pt,           % 程式碼與邊框距離
    numbersep=8pt,          % 行號與程式碼距離
    fontsize=\scriptsize,   % 字體大小
    breaklines,             % 自動換行
    tabsize=4,              % tab 寬度
    rulecolor=\color{black},% 框線顏色
    xleftmargin=1.5em       % 左側縮排
}

\title{Nested Loop, Arrays}
\author{Tai, Wei Hsuan}
\date{week 5}

\begin{document}
	\begin{frame}
		\titlepage
	\end{frame}

\begin{frame}
    \frametitle{Announcement}
    Recording or photographing the slides or any content displayed on the screen is permitted for personal use only. \\[6pt]
    Redistribution, modification, or any commercial use of the materials is strictly prohibited. \\[12pt]
    \textit{© 2025 [Tai, Wei Hsuan]. All rights reserved. For personal use only.}
\end{frame}


    % course material qrcode
    \begin{frame}
        \frametitle{\href{https://drive.google.com/drive/folders/14Tkn-rddw0k1obeOxkWi00S43M0e9wlW?usp=sharing}{Course Materials}}
        \begin{figure}
            \centering
            \includegraphics[width=0.6\textwidth]{src/qrcode.png}
        \end{figure}
    \end{frame}

    \begin{frame}
        \frametitle{進度安排}
        \makebox[\linewidth][c]{%
            \begin{tabular}{cl cl}
            \toprule
            日期 & 主題 & 日期 & 主題 \\
            \midrule
            9/12  & 課程簡介、基礎輸入輸出、變數   & 11/28 & 函式 \\
            9/19  & 變數的運算           & 12/5  & 遞迴 \\
            10/17 & 選擇結構與邏輯運算子     & 12/12 & Struct, Vector\\
            10/31 & 重複結構                 & 12/19 & Stack, Queue \\
            11/7  & 巢狀迴圈、陣列                 & 12/26 & TBD \\
            11/21 & 期中考                   & 1/2   & TBD \\
                &                          & 1/9   & 期末考 \\
            \bottomrule
            \end{tabular}
        }
    \end{frame}

    \begin{frame}
        \frametitle{Outline}
        \tableofcontents
    \end{frame}

    \section{Recap}
    \begin{frame}[fragile]
        \frametitle{Basic Structure}
        \begin{minted}{c++}
            #include<bits/stdc++.h>
            using namespace std;

            int main(){
                // code here
                return 0;
            }            
        \end{minted}
    \end{frame}
    \begin{frame}[fragile]
        \frametitle{Variables and basic I/O}
        \begin{minted}{c++}
            #include<bits/stdc++.h>
            using namespace std;

            int main(){
                int a;
                cin >>a;
                cout<<"The value of a is: "<<a<<'\n';
            }
        \end{minted}
    \end{frame}
    \begin{frame}[fragile]
        \frametitle{Selection Statements}
        \begin{minted}{c++}
            if(condition){
                // code when condition is true
            }else if(condition2){
                // code when condition2 is true
            }else{
                // code when both conditions are false
            }
        \end{minted}
    \end{frame}
    \begin{frame}[fragile]
        \frametitle{for loop}
        How to utilize \texttt{for} loop to print odd numbers from 1 to n?
        \begin{minted}{c++}
            int n;
            cin >>n;
            for(int i=1;i<=n;i++){
                if(i%2==1){
                    cout<<i<<'\n';
                }
            }
        \end{minted}
    \end{frame}
    \begin{frame}[fragile]
        \frametitle{for loop(cont.)}
        You can also:
        \begin{minted}{c++}
            int n;
            cin >>n;
            for(int i=1;i<=n;i+=2){
                cout<<i<<'\n';
            }
        \end{minted}
        Question: Can you use \texttt{continue} to achieve the same effect?
    \end{frame}
    \begin{frame}[fragile]
        \frametitle{while loop}
        \begin{minted}{c++}
            int i=1, n;
            cin >>n;         
            while(i<=n){
                cout<<i<<'\n';
                i+=2;
            }   
        \end{minted}
    \end{frame}
    \begin{frame}
        \frametitle{Prime Number Detection}
        Given a positive integer n, how to determine whether it is a prime number?
        
        It is well known that a prime number is a natural number greater than 1 that cannot be formed by multiplying two smaller natural numbers. In other words, a prime number has exactly two distinct positive divisors: 1 and itself.

        Thus, we can just check whether n is divisible by any integer from 2 to n-1.
    \end{frame}
    \begin{frame}[fragile]
        \frametitle{Implementation}
        \begin{minted}{c++}
            int n;
            cin >>n;
            bool is_prime=true;
            for(int i=2;i<n;i++){
                if(n%i==0){
                    is_prime=false;
                    break;
                }
            }
            if(is_prime)cout<<n<<" is a prime number.\n";
            else cout<<n<<" is not a prime number.\n";
        \end{minted}
        The above implementation has a small error. Can you spot it?
    \end{frame}
    \begin{frame}[fragile]
        \frametitle{Little Error}
        What if the input n is 1? By definition, 1 is not a prime number. However, the above code will output that 1 is a prime number.
        To fix this, we can add a condition to check if n is less than or equal to 1.
        \begin{minted}{c++}
            int n;
            cin >>n;
            bool is_prime=true;
            if(n<=1)is_prime=false;
            for(int i=2;i<n;i++){
                if(n%i==0){
                    is_prime=false;
                    break;
                }
            }
            if(is_prime)cout<<n<<" is a prime number.\n";
            else cout<<n<<" is not a prime number.\n";
        \end{minted}
    \end{frame}
    \begin{frame}
        \frametitle{Optimization}
        A computer can do about $10^8$ operations per second. If n is very large (e.g., $10^{12}$), the above algorithm may take too long to run.

        Note that if n is divisible by a number larger than $\sqrt{n}$, the corresponding divisor must be smaller than $\sqrt{n}$. Therefore, we only need to check for divisibility up to $\sqrt{n}$.

        With this optimization, we can efficiently determine the primality of large numbers.
    \end{frame}
    \begin{frame}[fragile]
        \frametitle{Implementation}
        \begin{minted}{c++}
            int n;
            cin >>n;
            bool is_prime=true;
            if(n<=1)is_prime=false;
            for(int i=2;i*i<=n;i++){
                if(n%i==0){
                    is_prime=false;
                    break;
                }
            }
            if(is_prime)cout<<n<<" is a prime number.\n";
            else cout<<n<<" is not a prime number.\n";
        \end{minted}
    \end{frame}    
    \begin{frame}
        \frametitle{Scope of the variables}
        The scope of a variable is the region of the program where the variable is defined and can be accessed. In C++, the scope of a variable is determined by where it is declared. Variables declared inside a block (enclosed by curly braces \{\}) are local to that block and cannot be accessed outside of it. Variables declared outside of any block are global and can be accessed from anywhere in the program.

        If you declare a variable globally, the variable will automatically be initialized to zero. However, if you declare a variable locally, the variable will have a random value if not initialized.
    \end{frame}

    \section{Nested Loop}
    \begin{frame}
        \frametitle{Nested Loop}
        The main idea is, anything can be inserted into a loop, including another loop. If we want to do some loop repeatedly, we can use nested loops.

        For instance, how to print a $5\times 4$ rectangle of asterisks(*)? Note that there are 5 rows and 4 columns.
    \end{frame}
    \begin{frame}[fragile]
        \frametitle{Implementation}
        \begin{minted}{c++}
            for(int i=1;i<=5;i++){          
                for(int j=1;j<=4;j++){     
                    cout<<"*";
                }
                cout<<'\n';                  
            }
        \end{minted}
    \end{frame}
    \begin{frame}[fragile]
        \frametitle{Another example}
        Note that in the inner loop, the veiables defined in the outer loop are also accessible. Thus, we can use nested loops to print a multiplication table.
        \begin{minted}{c++}
            for(int i=1;i<=9;i++){  
                for(int j=1;j<=9;j++){   
                    cout<<i<<"x"<<j<<"="<<i*j<<'\t';
                }
                cout<<'\n';                  
            }
        \end{minted}
    \end{frame}
    \begin{frame}
        \frametitle{Practice}
        ZeroJudge Problem c418: \href{https://zerojudge.tw/ShowProblem?problemid=c418}{c418. Bert的三角形 (1)}
    \end{frame}

    \section{Arrays}
    \begin{frame}
        \frametitle{Basic Concept}
        If you regard variables as boxes that can store a single value, then an array is like a row of boxes that can store multiple values of the same type. Each box in the array can be accessed using an index, which indicates its position in the array. In fact, arrays are contiguous memory locations that store multiple values of the same type.
    \end{frame}
    \begin{frame}
        \frametitle{Declare an Array}
        In the previous page, we know that an array is a collection of variables of the same type. To declare an array, we need to specify the type of the elements and the number of elements in the array. The syntax is similar to variables declaration, as follows:\texttt{TYPE IDENTIFIER[SIZE]}

        For instance, \texttt{int a[10]} declares an array of 10 integers.
    \end{frame}
    \begin{frame}
        \frametitle{Usage of Arrays}
        You can consider each element in the array as a separate variable. You can access and modify each element using its index. Note that the index starts from 0. For example, to access the first element of the array \texttt{a}, you can use \texttt{a[0]}. To access the second element, you can use \texttt{a[1]}, and so on.
    \end{frame}
    \begin{frame}
        \frametitle{Initialize an Array}
        As variables, the values of the elements in an array are also random if not initialized. Thus, we often initialize an array when declaring it. The simplest way to initialize an array is to set all elements to zero. We can do this by adding \texttt{=\{0\}} after the declaration. For example, \texttt{int a[10]=\{0\}} declares an array of 10 integers and initializes all elements to zero.

        Declaring an array globally will also initialize all elements to zero automatically.
    \end{frame}
    \begin{frame}
        \frametitle{Notes of Arrays}
        \begin{itemize}
            \item The size of the array is fixed while declaring it. You cannot change the size of the array after declaring it.
            \item Using a variable as the size of the array is illegal. The size must be a constant integer. Some compilers may allow it as an extension, but it is not standard C++ (i.e., not portable).
            \item The maximum size of an array is determined by the amount of memory available. The limit of a global array is about several million elements, while the limit of a local array is about a few tens of thousands of elements. 
        \end{itemize}
    \end{frame}

    \begin{frame}[fragile]
        \frametitle{Some examples}
        \begin{minted}{c++}
            int a[5];
            for(int i=0;i<5;++i){
                cin>>a[i];
            }
            for(int i=0;i<5;++i){
                cout<<a[i]<<" ";
            }
        \end{minted}
    \end{frame}
    \begin{frame}[fragile]
        \frametitle{String}
        Sometimes, the format of the input is a string of characters. In C++, we can use a character array to store a string. However, a character array is inconvenient to use. Thus, C++ provides a built-in string type that is easier to use. Here is an example of using a string:
        \begin{minted}{c++}
            string s;
            cin >>s;
            cout<<"The input string is: "<<s<<'\n';
            cout<<"The length of the string is: "<<s.size()<<'\n';
            cout<<"The first character of the string is: "<<s[0]<<'\n';
        \end{minted}
    \end{frame}
    \begin{frame}
        \frametitle{String(cont.)}
        Note that the index of a string also starts from 0. Thus, the first character of the string is at index 0, the second character is at index 1, and so on. You can also use a loop to iterate through each character in the string.

        You can consider a string as a flexible array of characters that can grow or shrink in size as needed. The usage of strings is similar to arrays, but strings have additional functionalities that make them easier to work with. If you truly interest in string, you can refer to \href{https://cplusplus.com/reference/string/string/}{C++ String Reference} for more details.
    \end{frame}

    \begin{frame}
        \frametitle{Practice}
        \begin{enumerate}
            \item Given n numbers, list them in reverse order.
            \item Given n numbers, list the frequency of each number, each number is between 1 and 100.
            \item The formula of standard deviation is:$SD=\sqrt{\frac{1}{n}\sum_{i=1}^{n}(x_i-\mu)^2}$, where $\mu$ is the mean of the n numbers. Given n numbers, calculate the standard deviation.
            \item Given n numbers, find the length of the longest continuous increasing subsequence.
            \item Given n numbers, find their greatest common divisor (GCD) and least common multiple (LCM).
            \item Given n numbers, find the maximum sum of any contiguous subarray (elements in the array may be negative).
        \end{enumerate}
    \end{frame}
\end{document}