\documentclass[xcolor=dvipsnames]{beamer}

% ==== Beamer 主題與配色 ====
\usetheme{CambridgeUS}
% \useoutertheme{miniframes}
% \useinnertheme{circles}

% \definecolor{UBCblue}{rgb}{0.04706, 0.13725, 0.26667}
% \definecolor{UBCgrey}{rgb}{0.3686, 0.5255, 0.6235}

% \setbeamercolor{palette primary}{bg=UBCblue,fg=white}
% \setbeamercolor{palette secondary}{bg=UBCblue,fg=white}
% \setbeamercolor{palette tertiary}{bg=UBCblue,fg=white}
% \setbeamercolor{palette quaternary}{bg=UBCblue,fg=white}
% \setbeamercolor{structure}{fg=UBCblue}
% \setbeamercolor{section in toc}{fg=UBCblue}
% \setbeamercolor{subsection in head/foot}{bg=UBCgrey,fg=white}

% ==== 語言與字型設定 ====
\usepackage{fontspec}
\usepackage{xeCJK}

% 英文與中文字型

\setCJKmainfont{Noto Serif CJK TC}[Script=CJK]
\setCJKmonofont{Noto Sans Mono CJK TC}[Script=CJK]

% ==== 套件 ====
\usepackage{amsmath, amssymb}
\usepackage{graphicx}
\usepackage{hyperref}
\usepackage{xcolor}
\usepackage{listings}
\usepackage{minted}
\usepackage{hyperref}
\usepackage{fvextra}

% ==== C 語言樣式 ====
\lstdefinestyle{C}{
    language=C,
    basicstyle=\ttfamily\bfseries\small,
    numbers=left,
    numbersep=5pt,
    tabsize=4,
    frame=single,
    commentstyle=\itshape\color{brown},
    keywordstyle=\bfseries\color{blue},
    deletekeywords={define},
    morekeywords={NULL, bool}
}

\title{Basic Skills 1}
\author{TAI, WEI HSUAN}
\date{\today}

\begin{document}
	
	\begin{frame}
		\titlepage
	\end{frame}
	
	\begin{frame}
		\frametitle{Outline}
        \begin{itemize}
            \item Sorting Algorithms
            \item Binary Search
            \item Fast Exponentiation
            \item Prefix Sum
        \end{itemize}
	\end{frame}
	
    \begin{frame}
        \frametitle{Sorting Algorithms}
        Given a series of numbers, can you sort them in ascending order?

        This is a very common problem in programming,
        and there are many algorithms to solve this problem.

        There are many sorting algorithms, such as:
        \begin{itemize}
            \item Bubble Sort
            \item Selection Sort
            \item Insertion Sort
            \item Merge Sort
            \item Quick Sort
            \item Heap Sort
            \item Counting Sort
            \item Radix Sort
            \item Bucket Sort
        \end{itemize}
    \end{frame}

    \begin{frame}[fragile]
        \frametitle{Sorting Algorithms}
        In this class, we won't go into details about each sorting algorithm,
        instead, we will use the built-in sorting function in C++.
    
        If you want to learn more about sorting algorithms,
        you can check out this link: \url{https://oi-wiki.org/basic/sort-intro/}

        The built-in sorting function is based on the \texttt{std::sort} function
    \end{frame}

	\begin{frame}[fragile]
        \frametitle{Code for Sorting Algorithms}
        \begin{minted}[
            linenos,                      % 開啟行號
            frame=lines,                  % 上下框線
            framesep=5pt,                 % 程式碼與邊框距離
            numbersep=8pt,                % 行號與程式碼距離
            fontsize=\scriptsize,        % 可視需求調整字體大小
            breaklines,                  % 長行自動換行(建議)
            tabsize=4,                   % tab 寬度
            rulecolor=\color{black},     % 框線顏色
            xleftmargin=1.5em            % 讓行號有空間顯示 
        ]{cpp}
        #include<algorithm>
        int arr[10005], n;

        sort(arr, arr+n); // Sort the array in ascending order

        \end{minted}
        Using this function, we can sort an array in ascending order with the time complexity of $O(n \log n)$,
        where $n$ is the number of elements in the array.
	\end{frame}

    \begin{frame}[fragile]
        \frametitle{Another way to sort}
        If you want to sort an array in your own way,
        you can define a custom comparator function.

        For example, if you want to sort an array in descending order,
        you can define a function like this:
        \begin{minted}[
            linenos,                      % 開啟行號
            frame=lines,                  % 上下框線
            framesep=5pt,                 % 程式碼與邊框距離
            numbersep=8pt,                % 行號與程式碼距離
            fontsize=\scriptsize,        % 可視需求調整字體大小
            breaklines,                  % 長行自動換行(建議)
            tabsize=4,                   % tab 寬度
            rulecolor=\color{black},     % 框線顏色
            xleftmargin=1.5em            % 讓行號有空間顯示 
        ]{cpp}
        bool cmp(int a, int b) {
            return a>b; // Return true if a is greater than b
        }

        sort(arr, arr+n, cmp); // Sort the array in descending order

        // Or you can use a lambda function
        sort(arr, arr+n, [](int a, int b) {
            return a > b;
        });
        \end{minted}

    \end{frame}

    \begin{frame}
        \frametitle{Principle of Comparison function}
        The comparison function should return true if the first argument 
        should come before the second in the final sorted order; 
        otherwise, it should return false.
    \end{frame}

    \begin{frame}
        \frametitle{More about sorting}
        
        Give you a series of student scores, sort them by scores.
        If scores are the same, the sort by student ID in ascending order.

    \end{frame}
    \begin{frame}[fragile]
        \frametitle{More about sorting}
        \begin{minted}[
            linenos,                      % 開啟行號
            frame=lines,                  % 上下框線
            framesep=5pt,                 % 程式碼與邊框距離
            numbersep=8pt,                % 行號與程式碼距離
            fontsize=\scriptsize,        % 可視需求調整字體大小
            breaklines,                  % 長行自動換行(建議)
            tabsize=4,                   % tab 寬度
            rulecolor=\color{black},     % 框線顏色
            xleftmargin=1.5em            % 讓行號有空間顯示 
        ]{cpp}
        struct Student{
            int id; // Student ID
            int score; // Student score
        };
        bool cmp(Student a, Student b) {
            if(a.score==b.score)return a.id<b.id; // If scores are the same, sort by ID
            return a.score>b.score; // Sort by score in descending order
        }
        vector<Student> students;
        sort(students.begin(), students.end(), cmp); // Sort the students by score and ID
        \end{minted}

    \end{frame}

	\begin{frame}[fragile]
        \frametitle{Binary Search}
        Given a sorted array, can you find the position of a target value in the array?

        Ofcourse, we can use a linear search to do that.
        \begin{minted}[
            linenos,                      % 開啟行號
            frame=lines,                  % 上下框線
            framesep=5pt,                 % 程式碼與邊框距離
            numbersep=8pt,                % 行號與程式碼距離
            fontsize=\scriptsize,        % 可視需求調整字體大小
            breaklines,                  % 長行自動換行(建議)
            tabsize=4,                   % tab 寬度
            rulecolor=\color{black},     % 框線顏色
            xleftmargin=1.5em            % 讓行號有空間顯示 
        ]{cpp}
        int linear_search(vector<int> arr, int target) {
            for(int i=0;i<arr.size();i++) {
                if(arr[i]==target) {
                    return i; // Found the target at index i
                }
            }
            return -1; // Target not found
        }
        \end{minted}
        Time complexity?
	\end{frame}

	\begin{frame}[fragile]
        \frametitle{Binary Search}
        We can use a concept of "cut in half"

        For example, if you want to gurss a number between 1 and 100,
        you can start by guessing 50, if the answer is smaller than 50, you can cut the range to 1-49,
        if the answer is larger than 50, you can cut the range to 51-100.
	\end{frame}

	\begin{frame}[fragile]
        \frametitle{Code for Binary Search}

        \begin{minted}[
            linenos,                      % 開啟行號
            frame=lines,                  % 上下框線
            framesep=5pt,                 % 程式碼與邊框距離
            numbersep=8pt,                % 行號與程式碼距離
            fontsize=\scriptsize,        % 可視需求調整字體大小
            breaklines,                  % 長行自動換行(建議)
            tabsize=4,                   % tab 寬度
            rulecolor=\color{black},     % 框線顏色
            xleftmargin=1.5em            % 讓行號有空間顯示 
        ]{cpp}
        vector<int> arr;
        int target;
        int binary_search(int l, int r){
            if(l>r)return -1; // Not found
            int mid=l+r>>1;
            if(mid==target)return mid; // Found the target
            if(arr[mid]>target)return binary_search(l, mid-1); // Search in the left half
            return binary_search(mid+1, r); // Search in the right half
        }
        \end{minted}
        Time complexity?
	\end{frame}

    \begin{frame}
        \frametitle{Practice}
        \begin{itemize}
            \item \href{https://www.spoj.com/problems/BSEARCH1/en/}{SPOJ - Binary Search}
        \end{itemize}
    \end{frame}
        

    \scriptsize
    \begin{frame}[fragile]
        \frametitle{C++ have a built-in function for binary search}
        \begin{minted}[
            linenos,                      % 開啟行號
            frame=lines,                  % 上下框線
            framesep=5pt,                 % 程式碼與邊框距離
            numbersep=8pt,                % 行號與程式碼距離
            fontsize=\scriptsize,        % 可視需求調整字體大小
            breaklines,                  % 長行自動換行(建議)
            tabsize=4,                   % tab 寬度
            rulecolor=\color{black},     % 框線顏色
            xleftmargin=1.5em            % 讓行號有空間顯示 
        ]{cpp}
            #include<algorithm>
            vector<int> arr;
            int target;
            int idnex=lower_bound(arr.begin(), arr.end(), target)-arr.begin();

        \end{minted}
    \end{frame}


    \begin{frame}
        \frametitle{Summary}
        \begin{itemize}
            \item We can use binary search to find the position of a target value in a sorted array 
                with the time complexity of $O(\log n)$, 
                where $n$ is the number of elements in the array.
            \item C++ have a built-in function for binary search,
                which is \texttt{lower\_bound} and \texttt{upper\_bound}.
        \end{itemize}

        
    \end{frame}

    % fast exponentiation
    \begin{frame}[fragile]
        \frametitle{Fast Exponentiation}
        Given a number $a$ and an exponent $b$,
        can you calculate $a^b$ ?

        Ofcourse we can use a linear search to do that.
        \begin{minted}[
            linenos,                      % 開啟行號
            frame=lines,                  % 上下框線
            framesep=5pt,                 % 程式碼與邊框距離
            numbersep=8pt,                % 行號與程式碼距離
            fontsize=\scriptsize,        % 可視需求調整字體大小
            breaklines,                  % 長行自動換行(建議)
            tabsize=4,                   % tab 寬度
            rulecolor=\color{black},     % 框線顏色
            xleftmargin=1.5em            % 讓行號有空間顯示 
        ]{cpp}
            int a, b, ans=1, mod=1e9+7;
            for(int i=0;i<b;i++){
                ans=ans*a%mod; // Multiply a to ans, and take mod
            }
            cout<<ans; // Output the result
        \end{minted}
        Time complexity?
    \end{frame}
    
    \begin{frame}
        \frametitle{Fast Exponentiation}
        Fast exponentiation is a technique to compute large powers of a number efficiently.
        The idea is to use the property:
        \[
            a^b = \begin{cases}
                1 & \text{if } b = 0 \\
                a \cdot a^{b-1} & \text{if } b \text{ is odd} \\
                (a^{b/2})^2 & \text{if } b \text{ is even}
            \end{cases}
        \]
    \end{frame}

    \begin{frame}
        \frametitle{Example}
        If you want to calculate $2^{15}$, you can do it like this:
        \begin{itemize}
            \item $2^{15} = 2^{14} \cdot 2$
            \item $2^{14} = (2^7)^2$
            \item $2^7 = 2^6 \cdot 2$
            \item $2^6 = (2^3)^2$
            \item $2^3 = 2^2 \cdot 2$
            \item $2^2 = (2^1)^2$
            \item $2^1 = 2$
        \end{itemize}
        This way, we can reduce the number of multiplications needed to calculate $2^{15}$.

        To calculate $a^b$, we can reduce the time complexity to $O(\log b)$
        by using the fast exponentiation technique.
   \end{frame}

    \begin{frame}[fragile]
        \frametitle{Code for fast exponentiation}
        \begin{minted}[
            linenos,                      % 開啟行號
            frame=lines,                  % 上下框線
            framesep=5pt,                 % 程式碼與邊框距離
            numbersep=8pt,                % 行號與程式碼距離
            fontsize=\scriptsize,        % 可視需求調整字體大小
            breaklines,                  % 長行自動換行(建議)
            tabsize=4,                   % tab 寬度
            rulecolor=\color{black},     % 框線顏色
            xleftmargin=1.5em            % 讓行號有空間顯示 
        ]{cpp}
        int fpow(int a, int b){
            int ans=1;
            while(b>0){
                if(b & 1)ans=ans*a%mod; // If b is odd, multiply a to ans
                a=a*a%mod; // Square a
                b>>=1; // Divide b by 2
            }
            return ans;
        }
        \end{minted}
        
    \end{frame}

    \begin{frame}
        \frametitle{Practice}
        \begin{itemize}
            \item \href{https://cses.fi/problemset/task/1095}{CSES 1095 - Pure Exponentiation}
        \end{itemize}
    \end{frame}

    \begin{frame}[fragile]
        \frametitle{Prefix Sum}
        Given an array with $n$ elements adn $q$ queries,
        can you calculate the sum of elements in a range $[l, r]$?

        One simple way is to use a loop to calculate the sum.
        \begin{minted}[
            linenos,                      % 開啟行號
            frame=lines,                  % 上下框線
            framesep=5pt,                 % 程式碼與邊框距離
            numbersep=8pt,                % 行號與程式碼距離
            fontsize=\scriptsize,        % 可視需求調整字體大小
            breaklines,                  % 長行自動換行(建議)
            tabsize=4,                   % tab 寬度
            rulecolor=\color{black},     % 框線顏色
            xleftmargin=1.5em            % 讓行號有空間顯示 
        ]{cpp}
        vector<int> arr;
        int q, l, r;
        for(int i=0;i<q;i++){
            cin>>l>>r;
            int sum=0;
            for(int j=l;j<=r;j++){
                sum+=arr[j];
            }
            cout<<sum<<"\n";
        }
        \end{minted}        
        Time complexity?
    \end{frame}

    \begin{frame}
        \frametitle{Prefix Sum}
        We can use a technique called prefix sum to solve this problem.

        Set $$p_i=\displaystyle\sum_{k=1}^{i}a_k$$

        If we want to calculate the sum of elements in a range $[l, r]$,
        we can use the formula:
        $$\displaystyle\sum_{k=l}^{r}a_k = \displaystyle\sum_{k=1}^{r}a_k - \displaystyle\sum_{k=1}^{l-1}a_k$$

        Though we need $O(n)$ to preprocess the prefix sum array, we can reduce the time complexity to $O(1)$ for each query.
    \end{frame}

    \begin{frame}[fragile]
        \frametitle{Code for Prefix Sum}
        \begin{minted}[
            linenos,                      % 開啟行號
            frame=lines,                  % 上下框線
            framesep=5pt,                 % 程式碼與邊框距離
            numbersep=8pt,                % 行號與程式碼距離
            fontsize=\scriptsize,        % 可視需求調整字體大小
            breaklines,                  % 長行自動換行(建議)
            tabsize=4,                   % tab 寬度
            rulecolor=\color{black},     % 框線顏色
            xleftmargin=1.5em            % 讓行號有空間顯示 
        ]{cpp}
        vector<int> arr;
        int pre[N], q, l, r;
        for(int i=1;i<=n;i++){
            pre[i]=pre[i-1]+arr[i]; // Calculate the prefix sum
        }
        for(int i=0;i<q;i++){
            cin>>l>>r;
            cout<<pre[r]-pre[l-1]<<"\n"; // Output the sum of elements in the range [l, r]
        }
        \end{minted}        
        
    \end{frame}    

    \begin{frame}
        \frametitle{Challenge}
        Given an array with $n$ elements and $q$ queries, each query has one number.
        Can you output a pair of indices $(l, r)$ such that the sum of elements in the range 
        $[l, r]$ is equal to the number in the query?

    \end{frame}

    \begin{frame}
        \frametitle{Practice}
        \begin{itemize}
            \item \href{https://cses.fi/problemset/task/1646}{CSES 1646 - Pure Prefix Sum}
        \end{itemize}
    \end{frame}
\end{document}