\documentclass[12pt,a4paper]{article}
\usepackage[margin=2cm]{geometry}
\usepackage{xeCJK}
\usepackage{fontspec}
\setCJKmainfont{Noto Serif CJK TC}[Script=CJK]
\usepackage{amsmath,amssymb}
\usepackage{graphicx}
\usepackage{fancyhdr}
\setlength{\headheight}{14.5pt}
\addtolength{\topmargin}{-2.5pt}
\usepackage{hyperref}
\usepackage{listings}
\usepackage{enumitem}
\usepackage{titlesec}
\usepackage{caption}
\usepackage{indentfirst}
\usepackage{booktabs}
\usepackage{longtable}
\usepackage{multirow}
\usepackage{array}
\usepackage{tabularx}
\usepackage{float}
\usepackage{minted}
\setlength{\parindent}{2em}
\pagestyle{fancy}
\fancyhf{}
\cfoot{\thepage}
\linespread{1.3}
\setminted{
    linenos,                % 行號
    frame=lines,            % 上下框線
    framesep=5pt,           % 程式碼與邊框距離
    numbersep=8pt,          % 行號與程式碼距離
    fontsize=\scriptsize,   % 字體大小
    breaklines,             % 自動換行
    tabsize=4,              % tab 寬度
    rulecolor=\color{black},% 框線顏色
    xleftmargin=1.5em       % 左側縮排
}

\title{Week 5 Reference Solutions}
\author{Tai, Wei-Hsuan}
\date{\today}

\begin{document}

\maketitle

\lhead{Week 5 Solutions}
\rhead{Tai, Wei-Hsuan}

\begin{enumerate}
    \item Given n numbers, list them in reverse order.
    \begin{itemize}
        \item Idea: Just output the array from the last element to the first element.
        \item Sample Input:
        \begin{minted}{text}
5
10 20 30 40 50
        \end{minted}
        \item Sample Output:
        \begin{minted}{text}
50 40 30 20 10
        \end{minted}
        \item Solution:
        \begin{minted}{cpp}
#include<bits/stdc++.h>
using namespace std;

int main(){
    int arr[10005], n;
    cin >>n;
    for(int i=0;i<n;++i){
        cin >>arr[i];
    }
    for(int i=n-1;i>=0;--i){
        cout<<arr[i]<<' ';
    }
}
        \end{minted}
    \end{itemize}
    \item Given n numbers, list the frequency of each number.
    \begin{itemize}
        \item Idea: Use an array to count the frequency of each number.
        \item Sample Input:
        \begin{minted}{text}
7
1 1 2 2 3 3 3 3
        \end{minted}
        \item Sample Output:
        \begin{minted}{text}
1:2
2:2
3:3
        \end{minted}
        \item Solution:
        \begin{minted}{c++}
#include<bits/stdc++.h>
using namespace std;

int main(){
    int arr[105]={0}, n, mx=-1;
    cin >>n;
    for(int i=0;i<n;++i){
        int tmp;
        cin >>tmp;
        mx=max(mx, tmp);
        ++arr[tmp];
    }   
    for(int i=0;i<=mx;++i){
        if(arr[i])cout<<i<<":"<<arr[i]<<'\n';
    }
}
        \end{minted}
    \end{itemize}
    \item The formula of standard deviation is:$SD=\sqrt{\frac{1}{n}\sum_{i=1}^{n}(x_i-\mu)^2}$, where $\mu$ is the mean of the n numbers. Given n numbers, calculate the standard deviation.
    \begin{itemize}
        \item Idea: First calculate the mean, then calculate the standard deviation using the formula.
        \item Sample Input:
        \begin{minted}{text}
5
1 2 3 4 5
        \end{minted}
        \item Sample Output:
        \begin{minted}{text}
1.41421
        \end{minted}
        \item Solution:
        \begin{minted}{c++}
#include<bits/stdc++.h>
using namespace std;

int main(){
    int arr[10005], mean=0, n;
    
    cin >>n;
    for(int i=0;i<n;++i){
        cin >>arr[i];
        mean+=arr[i];
    }
    mean/=n;

    double sd=0;
    for(int i=0;i<n;++i){
        sd+=(arr[i]-mean)*(arr[i]-mean);
    }
    sd=sqrt(sd/n);

    cout<<sd;
}
        \end{minted}
    \end{itemize}
    \item Given n numbers, find the length of the longest continuous increasing subsequence.
    \begin{itemize}
        \item Idea: Enumerate the strat of the subsequence, and extend it as long as possible.
    \end{itemize}
\end{enumerate}

\end{document}