\documentclass[11pt,a4paper]{article}

\usepackage{indentfirst}
\usepackage{amssymb}
\usepackage{subcaption}
\usepackage{graphicx}
\usepackage{longtable}
\usepackage{fancyhdr}
\usepackage{xeCJK}
\usepackage{amsmath}
\usepackage{amssymb}
\usepackage{ulem}
\usepackage{xcolor}
\usepackage{fancyvrb}
\usepackage{listings}
\usepackage{soul}
\usepackage{hyperref}
	\lstdefinestyle{C}{
   		language=C, 
   		basicstyle=\ttfamily\bfseries,
    	numbers=left, 
    	numbersep=5pt,
    	tabsize=4,
    	frame=single,
   	 	commentstyle=\itshape\color{brown},
    	keywordstyle=\bfseries\color{blue},
   	 	deletekeywords={define},
    	morekeywords={NULL,bool}
	}	

\setCJKmainfont{Noto Serif CJK TC}[Script=CJK]
\setCJKmonofont{Noto Sans Mono CJK TC}[Script=CJK]
 
\voffset -20pt
\textwidth 410pt
\textheight 650pt
\oddsidemargin 20pt
\newcommand{\XOR}{\otimes}
\linespread{1.2}\selectfont

\pagestyle{fancy}
\lhead{114APCS進階班期中比賽}

\begin{document}

\begin{center}
\section*{C. Safe Pro}
\end{center}

\section*{Description}


東北一中銀行儲存了全東北一中的黃金,在銀行內部有一排編號為$1\sim 5000$的保險櫃,有許多客戶會將金條儲存在內,有時候也會提領金條,為避免有人盜領,每個保險庫都設有密碼。
YL是東北一中銀行的經理,由於銀行老闆實在是太摳了,為了減少薪水的支出,他壓榨員工致力於激發員工的個人工作能力,讓YL負責銀行所有的事務,但是保險櫃實在是太多了,用手記帳一定會記到死去,所以YL決定引進電腦來幫助他,可是他是一個電腦白癡,所以想請你幫他做一個可以記帳的程式。
喔對了,老闆每次都會隨機抽查保險庫裡面有多少金條,這讓YL壓力頗大,他需要你的幫助!!!


\section*{Input}

第一行為兩個正整數$M$和$N$,代表總共有$N$個客戶,並且他們只會存取編號$1\sim M$的保險庫

第二行為$M$個四位數,第$i$個字串$M_i$代表編號$i$保險庫的密碼

第三行為$M$個整數,第$i$個整數代表編號$i$保險庫一開始有多少金條

第四行為$N$個整數,第$i$個整數$a_i$代表第$i$個客戶想要開啟編號$a_i$的保險庫

第五行為$N$個數,第$i$個數$b_i$代表第i個客戶想要將$b_i$條金條存入編號$a_i$的保險庫

第六行為$N$個數,第$i$個數$b_i$代表第$i$個客戶想要從編號$a_i$的保險庫提領$b_i$條金條

第七行為$N$個四位數,第$i$個字串代表第$i$個客戶輸入的密碼

第八行為一個數字$K$,接下一行$K$個整數,每行一個整數$X$代表老闆抽查的保險庫編號,請輸出該保險庫有多少金條

請注意,如果密碼不正確是無法提領/存入金條的,另外,每個金庫的金條最少只能被提領至$0$條,如果提領後結果$<0$ YL最多只會讓該客戶提領到金條數量$=0$。你可以假設每一次操作都是先存入再提出。


\section*{Output}

輸出共一行,$K$個整數,第$K$個整數$K_i$代表編號$K_i$保險庫中的金條數量

\newpage
\section*{Sample 1}
\begin{longtable}[!h]{|p{0.5\textwidth}|p{0.5\textwidth}|}
\hline
\textbf {Input}	& \textbf {Output} \\
\hline
\parbox[t]{0.5\textwidth} % sample 1
{ \tt
3 4\\
1234 4321 9999\\
11 22 33 \\
1 2 3 2\\
3 3 3 3\\
2 4 2 4\\
1234 4321 7777 1324\\
5\\
1 2 3 2 1\\

} &
\parbox[t]{0.5\textwidth}
{ \tt
%output
12 21 33 21 12
} \\
\hline
\end{longtable}

\section*{Sample 2}
\begin{longtable}[!h]{|p{0.5\textwidth}|p{0.5\textwidth}|}
\hline
\textbf {Input}	& \textbf {Output} \\
\hline
\parbox[t]{0.5\textwidth} % sample 1
{ \tt
3 0\\
1234 4321 9999\\
11 22 33 \\
5\\
1 2 3 2 1\\


} &
\parbox[t]{0.5\textwidth}
{ \tt
%output
11 22 33 22 11
} \\
\hline
\end{longtable}

\section*{Subtasks}

在一個子任務的「測試資料範圍」的敘述中,如果存在沒有提到範圍的變數,則此變數的範圍為 Input 所描述的範圍。

\begin{center}
 \begin{tabular}{||c c c||} 
 \hline
 子任務編號 & 子任務配分 & 測試資料範圍 \\  
 \hline
 \hline
 1 & 20\% & $M<500, N=0$ \\ 
 \hline
 2 & 30\% & $M<500, N<20, K<50$,不會出現錯誤的密碼\\
 \hline 
 3 & 50\% & $M<5000, N<200, K<500$ \\
 \hline

\end{tabular}
\end{center}

配分說明:
- 保證密碼為四位數

\end{document}
