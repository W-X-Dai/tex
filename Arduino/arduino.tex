\documentclass[12pt,a4paper]{article}
\usepackage[margin=2cm]{geometry}
\usepackage{xeCJK}
\usepackage{fontspec}
\setCJKmainfont{Noto Serif CJK TC}[Script=CJK]
\setCJKmonofont{Noto Sans Mono CJK TC}[Script=CJK]
\usepackage{amsmath,amssymb}
\usepackage{graphicx}
\usepackage{fancyhdr}
\setlength{\headheight}{14.5pt}
\addtolength{\topmargin}{-2.5pt}
\usepackage{hyperref}
\usepackage{xcolor}
\usepackage{listings}
\usepackage{enumitem}
\usepackage{titlesec}
\usepackage{caption}
\usepackage{indentfirst}
\setlength{\parindent}{2em}
\pagestyle{fancy}
\fancyhf{}
\cfoot{\thepage}
\linespread{1.3}

\lstset{
  basicstyle=\ttfamily\footnotesize,  % 字型與大小
  keywordstyle=\color{blue},
  commentstyle=\color{gray},
  stringstyle=\color{orange},
  numbers=left,                       % 行號在左側
  numberstyle=\tiny\color{gray},
  stepnumber=1,                       % 每行都顯示行號
  numbersep=5pt,
  backgroundcolor=\color{white},
  frame=single,                       % 加上框線
  breaklines=true,                    % 長行自動換行
  tabsize=2,
  language=Python                     % 可以換成 C++, Java, etc.
}

\title{Arduino教學手冊}
\author{Tai, Wei-Hsuan}
\date{\today}

\begin{document}

\maketitle

\lhead{Arduino教學手冊}
\rhead{2025/07/05}
\newpage
\tableofcontents
\newpage

\newpage
\section{Arduino介紹}

平時你在寫程式時,會使用電腦的\texttt{CPU}來執行程式碼,
而\texttt{Arduino}的核心就是一顆微處理器,這顆微處理器可以執行你寫的程式碼,
並且控制外部的電子元件,例如LED燈、馬達、感測器等等。

\texttt{Arduino}的開發環境是\texttt{Arduino IDE},這是一個可以讓你寫程式、編譯程式、上傳程式到\texttt{Arduino}開發板的軟體,
我們已經幫你們安裝好了,但如果你在家裡想要嘗試,可以去\texttt{https://www.arduino.cc/en/software}下載最新版本的\texttt{Arduino IDE}。

\section{電子學的基本概念}

\subsection{電壓(Voltage)}
電壓是電流的驅動力,單位是伏特(Volt),簡稱V。電阻相同時,電壓越高,電流越大。

\subsection{電流(Current)}
電流是電荷的流動,單位是安培(A),簡稱I。電流越大,代表單位時間內通過的電荷量越多。

\subsection{電阻(Resistance)}
電阻是對電流流動的阻礙,單位是歐姆($\Omega$),簡稱R。在相同電壓下電阻越大,電流越小。

\subsection{歐姆定律(Ohm's Law)}
電壓、電流和電阻之間關係,可以使用歐姆定律來表示,公式為:
$$
V = I \cdot R
$$

這個公式告訴我們,當電壓不變時,電流和電阻成反比;當電阻不變時,電流和電壓成正比。

如果用水管來比喻,電壓就是水壓,電流就是水流的速度,電阻就是水管的阻塞程度,
水壓越大,水流的速度越快;水管阻塞程度越大,水壓就越大。

\section{如何透過\texttt{Arduino}控制元件}

當 \texttt{Arduino} 的腳位輸出電壓,並透過電路接到元件(例如 LED 或馬達)時,元件會因為電壓差而產生電流。這個電流流經元件內部,讓它發光、發聲或轉動。
這就是 \texttt{Arduino} 控制元件的基本原理。這一部份的內容會教你如何使用程式控制 \texttt{Arduino} 的腳位輸出電壓,並透過電路接到元件。

\subsection{Arduino IDE的程式結構}
Arduino IDE的程式結構主要由三個部分組成:設定區、主程式區和函式區。設定區用於初始化變數和設定腳位,主程式區則是執行的主要邏輯,而函式區則是用來定義可重複使用的函式。

以下的程式碼是你一打開Arduino IDE就會看到的範本程式碼:
\begin{lstlisting}
void setup() {
  // put your setup code here, to run once:
}
void loop() {
  // put your main code here, to run repeatedly:
}
\end{lstlisting}

其中,\texttt{setup()}函式是用來初始化變數和設定腳位的,這個函式只有在每次重新啟動時會執行一次,
而\texttt{loop()}函式則是用來執行主要邏輯的,這個函式會不斷重複執行,直到電源關閉或\texttt{Arduino}被重置。

一般而言,我們會在\texttt{setup()}函式中設定腳位的模式,例如輸入或輸出,然後在\texttt{loop()}函式中執行主要邏輯,例如讀取感測器的數據或控制元件的狀態。

你可以觀察到Arduino IDE的程式碼結構一樣是使用無數個函式來組成的,因此全域變數、自訂函式、類別等概念在Arduino IDE中也一樣適用,
但請記得考量到Arduino開法板的效能,盡量避免使用過多的全域變數與類別,這樣會影響程式的執行速度。

\subsection{\texttt{Arduino}的輸出}
平時使用電腦編譯\texttt{C++}程式時,會使用\texttt{cout}來輸出結果,
但在使用\texttt{Arduino}時,執行程式的核心不是電腦的\texttt{CPU},而是你手上那一塊\texttt{Arduino}開發板,
所以我們不能使用\texttt{cout}來輸出結果,因為\texttt{Arduino}開發板並沒有螢幕可以顯示結果。
而是使用\texttt{Serial}來輸出結果,這樣\texttt{Arduino}開發板就可以透過USB線將結果傳送到電腦上,
然後電腦就可以使用\texttt{Serial Monitor}來顯示結果。

這樣的作法等同於僅借用點腦的螢幕來顯示結果,電腦本身不參預運算。

\begin{lstlisting}
使用Serial輸出結果
Serial.begin(9600); //初始化Serial,9600是傳輸速率
Serial.println("Hello World!"); //輸出Hello World!並且換行
Serial.print("Hello World!"); //輸出Hello World!不換行
Serial.print(123); //輸出123
Serial.print(a); //輸出變數a的值
如果要串接變數與字串,可以這樣寫:
Serial.print("a的值是" + String(a)); //輸出a的值
\end{lstlisting}

\subsection{控制腳位}
假設你今天要讓一個\texttt{LED}燈亮起來,該怎麼做呢?你需要一個正極和一個負極,負極可以直接連接到開發板上的\texttt{GND}腳位,
而正極的操作就多了,開發板上有現成的\texttt{5V}腳位可以使用,但這個腳位是持續輸出電壓的,如果你想要控制他閃爍,就必須控制腳位的電壓輸出,
這時候就需要用到\texttt{digitalWrite()}這個函式了。
\begin{lstlisting}    
使用digitalWrite控制腳位

請記得先使用pinMode設定腳位的模式
void setup() {
  pinMode(13, OUTPUT); //設定腳位13為輸出模式
}

在loop()中做你想做的事情,例如以下程式碼會讓腳位13的電壓不斷地閃爍
void loop() {
    digitalWrite(13, HIGH); //將腳位13的電壓設為5V
    delay(1000); //延遲1秒(1000毫秒)
    digitalWrite(13, LOW); //將腳位13的電壓設為0V 
    delay(1000); //延遲1秒
    其中,HIGH也可以用1來表示,LOW也可以用0來表示,
    所以這兩行程式碼也可以寫成:
    digitalWrite(13, 1); //將腳位13的電壓設為5V
    delay(1000); //延遲1秒
    digitalWrite(13, 0); //將腳位13的電壓設為0V    
    delay(1000); //延遲1秒
}



\end{lstlisting}


\newpage
\section{Appendix- \texttt{C++}語法介紹}


\subsection{變數的種類與介紹}

\begin{table}[h!]
\centering
\begin{tabular}{|c||c|c|c|}
\hline
\textbf{類型} & \textbf{大小 (位元組)} & \textbf{範圍} & \textbf{用途} \\ \hline
\texttt{int} & 4 & -2,147,483,648 to 2,147,483,647 & 整數運算 \\ \hline
\texttt{double} & 8 & ±2.3E-308 to ±1.7E+308 & 浮點數運算 \\ \hline
\texttt{char} & 1 & -128 to 127 或 0 to 255 & 儲存單一字元 \\ \hline
\end{tabular}
\caption{\texttt{C++}常用的三種變數}
\label{tab:datatype_comparison}
\end{table}

雖然C++有以上三種常用變數,但在Arduino中,我們比較常使用的是\texttt{int}與\texttt{double}。要注意的是,由於Arduino的運算速度較慢,
使用\texttt{double}會影響運算速度。

以下是操縱變數的範例程式碼:


\begin{lstlisting}
宣告變數
int a=1, b=2;
double pi=3.14;

變數運算
int sum=a+b;//計算a+b
int mod=a%b;//計算a除以b的餘數

更改變數的值
a=3;//將變數a的值改為3
b+=a;//將變數b的值加上變數a的值

以下的寫法都是把變數a的值加1
a=a+1;
a+=1;
a++;
++a;
\end{lstlisting}
    
\subsection{if-else語法}
我們難免會遇到一些情況,必須根據不同的條件來執行不同的程式碼。這時候就需要用到\texttt{if-else}語法。


\begin{lstlisting}
if-else範例程式碼
if(a>b){
    //當a大於b時執行的程式碼
}else{
    //當a不大於b時執行的程式碼
}
\end{lstlisting}

在判別時,邏輯運算子也會派上用場,以下是常見的邏輯運算子:
\begin{table}[h!]
\centering
\begin{tabular}{|c|c|c|}
\hline
\textbf{運算子} & \textbf{描述} & \textbf{範例} \\ \hline
\texttt{\&\&(AND)}  & 當兩個條件都為真時,結果為真 & \texttt{(a > b) \&\& (b > c)} \\ \hline
\texttt{||(OR)}  & 當至少一個條件為真時,結果為真 & \texttt{(a > b) || (b > c)} \\ \hline
\end{tabular}
\caption{\texttt{C++}邏輯運算子}
\label{tab:logical_bitwise_operators}
\end{table}

\begin{lstlisting}
邏輯運算子範例程式碼
if(a>b && b>c){
    //當a大於b且b大於c時執行的程式碼
}else if(a>b || b>c){
    //當a大於b或b大於c時執行的程式碼
}else{
    //當a不大於b且b不大於c時執行的程式碼
}
\end{lstlisting}

\subsection{for loop}
如果要重複很多次執行同樣的程式碼,使用\texttt{for loop}會比重複寫一樣的程式碼來得簡單許多。
for loop的語法如下:
\begin{lstlisting}
for loop語法
for(初始值; 條件; 更新){
    //要執行的程式碼
}
舉例而言,假設我們要輸出1到10的數字,我們可以這樣寫:

for(int i=1;i<=10;i++){
    Serial.println(i);
}
其中,i是初始值,i<=10是條件,i++是更新的方式。
\end{lstlisting}

\subsection{while loop}
在明確知曉執行次數的情況下,使用\texttt{for loop}會比較簡單,
但在不明確知曉執行次數的情況下,使用\texttt{while loop}會比較直觀。
while loop的語法如下:
\begin{lstlisting}
while loop語法
while(條件){
    //要執行的程式碼
}
舉例而言,假設我們要輸出1到10的數字,我們可以這樣寫:
int i=1;
while(i<=10){
    Serial.println(i);
    i++;
}
在這個例子中,我們先宣告一個變數i,然後使用while loop來判斷i是否小於等於10,
如果是,就輸出i的值,然後將i的值加1,直到i大於10為止。
\end{lstlisting}
以上這是明確知曉執行次數的情況下,使用while loop的範例程式碼,明顯會比\texttt{for loop}來得繁瑣,
但在不明確知曉執行次數的情況下,使用while loop會比較直觀。
\begin{lstlisting}
String password="arduino123";
String input="";

while(input!=password) {
    Serial.println("請輸入密碼:");
    while(Serial.available()==0){
        //等待使用者輸入
    }
    input=Serial.readStringUntil('\n');
    input.trim();//去除換行符
}
Serial.println("密碼正確,歡迎進入系統!");
在這個例子中,我們使用while loop來判斷使用者輸入的密碼是否正確,
如果不正確,就一直要求使用者輸入,直到輸入正確為止。
\end{lstlisting}

\end{document}