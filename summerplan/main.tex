\documentclass[12pt,a4paper]{article}
\usepackage[margin=2cm]{geometry}
\usepackage{xeCJK}
\usepackage{fontspec}
\setCJKmainfont{Noto Serif CJK TC}[Script=CJK]
\usepackage{amsmath,amssymb}
\usepackage{graphicx}
\usepackage{fancyhdr}
\setlength{\headheight}{14.5pt}
\addtolength{\topmargin}{-2.5pt}
\usepackage{hyperref}
\usepackage{listings}
\usepackage{enumitem}
\usepackage{titlesec}
\usepackage{caption}
\usepackage{indentfirst}
\setlength{\parindent}{2em}
\pagestyle{fancy}
\fancyhf{}
\cfoot{\thepage}
\linespread{1.3}

\title{暑期專題計劃書\\\large 肺部電腦斷層掃描之非小細胞癌 PD-L1 表現預測:\\結合多任務自監督學習與生成對抗網路}
\author{申請者:戴偉璿}
\begin{document}

\rhead{2025 年暑期專題}
\lhead{肺部電腦斷層掃描之非小細胞癌 PD-L1 表現預測}

\maketitle

\newpage

\section{研究背景與動機}
PD-L1(Programmed Death Ligand 1)表現量是免疫治療中一個重要的生物標記,
常用以評估非小細胞肺癌(NSCLC)患者是否適合接受 PD-1/PD-L1 抑制劑。然而,
現有檢測方法依賴組織切片與免疫染色,具有侵入性、區域異質性與判讀主觀性等缺點。

因此,本研究旨在建構一套基於肺部電腦斷層掃描(CT)影像的 PD-L1 表現預測模型,
期望能以非侵入方式輔助臨床決策。本計畫以結合多任務自監督學習(MTMAE)與生成對抗網路(GAN)為主軸,
進一步探索三項潛在延伸方向以提升模型性能。

參照了周姵妤學姐的碩士論文:「肺部電腦斷層掃描之非小細胞癌 PD-L1 表現預測 :
結合遮蓋圖像模型與生成對抗網路」,其中提出了一種基於MAE模型改良後的多模態模型MTMAE,
結合了自監督重建、腫瘤分割與分類任務,並利用 GAN 強化訓練資料的多樣性,
在低資料條件下有效提升了 PD-L1 表現的預測準確率。基於該模型的潛力,
我想嘗試看看這個方法能否有進一步改進的空間。

\section{研究目標}
\begin{itemize}
  \item 建立以 MTMAE 為基礎之 PD-L1 表現預測模型
  \item 探討加入對比學習對自監督表徵學習的增強效果。
  \item 在ViT encoder 中嵌入 GNN,建構 patch 間關聯性以提升特徵整合能力。
  \item 評估多模型集成(ensemble)策略對預測穩定性與泛化能力的影響。
\end{itemize}

\section{研究方法}
本研究將基於周姵妤學姐的碩士論文所提出的 MTMAE 模型進行改良,
模型核心為 Multi-task Masked Autoencoder(MTMAE),包含,由於
影像資料的缺乏,我們先使用 GAΝ 生成批量的影像,對模型進行預訓練,接下來再使用
真正的醫學影像進行微調。

以下是我想到可以進行延伸的部份:

\begin{enumerate}
    \item 對比學習:在自監督訓練階段導入對比損失,以不同遮蔽策略產生的影像對作為正樣本,
    提升 encoder 對語意一致性的建模能力。
    \item GNN 結合:將 ViT encoder 輸出的 patch token 建構為圖結構,節點間依位置或注意力建邊,
    透過 GNN 進行訊息傳遞,強化區域語意整合。
    \item 模型集成:利用隨機初始化、遮蔽方式或 GAN 輸入生成多個 MTMAE 模型,最後透過投票的方式
    整合各模型的預測結果,提升整體穩定性與泛化能力。
\end{enumerate}

\section{實驗設計}

預計使用來自於台大醫院、台大醫院新竹分院、台大醫院雲林分院提供之非小細胞肺癌患者 CT 與 PD-L1 標記資料做為輸入,
比較原本的 MTMAE 模型、加入對比學習、加入 GNN 、集成模型在在
正確率(Accuracy)、靈敏度(Sensitivity)、  特異度(Specificity)與 AUC(Area under curve)
上的表現。

\section{預期成果}
本專題預期將建立可實作之 MTMAE 模型,完成 PD-L1 表現預測任務,
並驗證對比學習、GNN 以及模型集成等方法對於預測效能的提升效果。
期望透過這些改進能顯著提升模型的準確度與泛化能力。

\section{進度規劃(預估)}
\begin{tabular}{|c|c |l|  }
    \hline
    週次 & 日期 & 工作內容 \\
    \hline\hline
    1 & 6/16$\sim$6/22 & 閱讀論文、整理背景知識、建立 MTMAE 架構 \\
    \hline
    2$\sim$3 & 6/23$\sim$7/6 & 嘗試導入對比學習模組進行初步實驗 \\
    \hline
    4$\sim$5 & 7/7$\sim$7/20 & 嘗試加入 GNN 模組進行實驗 \\
    \hline
    6$\sim$7 & 7/21$\sim$8/3 & 嘗試集成模型進行實驗 \\
    \hline
    8 & 8/4$\sim$8/14 & 整理研究內容,撰寫書面報告 \\
    \hline
    9 & 8/15$\sim$8/21 & 完成最終簡報與展示準備 \\
    \hline
\end{tabular}

\end{document}