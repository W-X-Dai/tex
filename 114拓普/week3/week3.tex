\documentclass[xcolor=dvipsnames]{beamer}

% ==== 主題 ====
\usetheme{metropolis}
\usefonttheme{professionalfonts}          % 不覆蓋你自訂的字型

% ==== 字型 ====
\usepackage{fontspec}
\usepackage{xeCJK}
\renewcommand{\familydefault}{\rmdefault} % 使用 serif 字體(重點)

% 西文字型:Times New Roman 的開源替代品
\setmainfont{TeX Gyre Termes}[
  Ligatures=TeX,
  BoldFont={* Bold},
  ItalicFont={* Italic}
]

% 中文字型(可改為思源宋體、標楷體等)
\setCJKmainfont{Noto Serif CJK TC}
\setCJKsansfont{Noto Sans CJK TC} % 有需要再用
\setCJKmonofont{Noto Sans Mono CJK TC}

% ==== 數學字型(與正文字體一致)====
\usepackage{unicode-math}
\setmathfont{TeX Gyre Termes Math}

% ==== 套件 ====
\usepackage{amsmath, amssymb}
\usepackage{graphicx}
\usepackage{hyperref}
\usepackage{minted}
\usepackage{fvextra}
\usepackage{xcolor}
\usepackage{booktabs}

% ==== 顏色設定(可選)====
\definecolor{MyBlue}{RGB}{3, 55, 105}
\setbeamercolor{structure}{fg=MyBlue}
\setbeamercolor{block title}{bg=MyBlue,fg=white}
\setbeamercolor{block body}{bg=blue!5}
\setbeamertemplate{section in toc}{%
  \inserttocsectionnumber.~\inserttocsection\par
}

\setminted{
    linenos,                % 行號
    frame=lines,            % 上下框線
    framesep=5pt,           % 程式碼與邊框距離
    numbersep=8pt,          % 行號與程式碼距離
    fontsize=\scriptsize,   % 字體大小
    breaklines,             % 自動換行
    tabsize=4,              % tab 寬度
    rulecolor=\color{black},% 框線顏色
    xleftmargin=1.5em       % 左側縮排
}

\title{Selection Statements \& Logical Operators}
\author{Tai, Wei Hsuan}
\date{week 2}

\begin{document}
	\begin{frame}
		\titlepage
	\end{frame}


    % course material qrcode
    \begin{frame}
        \frametitle{\href{https://drive.google.com/drive/folders/14Tkn-rddw0k1obeOxkWi00S43M0e9wlW?usp=sharing}{Course Materials}}
        \begin{figure}
            \centering
            \includegraphics[width=0.7\textwidth]{src/qrcode.png}
        \end{figure}
    \end{frame}

    \begin{frame}
        \frametitle{Outline}
        \tableofcontents
    \end{frame}

    \section{Recap}
    \begin{frame}
        \frametitle{Remember}
        \begin{itemize}
            \item Every statement ends with a semicolon \textbf{(;)}.
            \item Programs always start from the \alert{main()} function.
            \item All functional code needs to be inside a function.
            \item All functions must be enclosed in curly braces \textbf{\{\}}.
        \end{itemize}
    \end{frame}
    \begin{frame}[fragile]
        \frametitle{Basic structure of C++}
        \begin{minted}{cpp}
            #include <bits/stdc++.h>
            using namespace std;

            int main() {
                // code here
                return 0;
            }
        \end{minted}
    \end{frame}
    \begin{frame}[fragile]
        \frametitle{Variables Declaration}
        \begin{minted}{cpp}
            int a;
            double b, pi=3.14;
            char c='A';
        \end{minted}
    \end{frame}
    \begin{frame}[fragile]
        \frametitle{Basic Input/Output}
        \begin{minted}{cpp}
            int a, b, c;
            cin >>a>>b>>c;
            cout<<"The result of a*10 is"<<a*10<<endl;
        \end{minted}
    \end{frame}
    \begin{frame}[fragile]
        \frametitle{What's the value of each variable?}
        \begin{minted}{cpp}
            int a, b;
            char c;
            //input: 10+5
            cin >>a>>c>>b;
        \end{minted}
    \end{frame}
    \begin{frame}
        \frametitle{A little tip}
        The result is \texttt{a=10, b=5, c='+'}\\
        In C++, the input stream will read the whole input until it meets a whitespace, newline or another data type.\\
        Therefore, \texttt{cin} will read \texttt{10} into \texttt{a}, then meet the character \texttt{'+'} and stop reading for \texttt{a}.\\
        Next, \texttt{cin} will read \texttt{'+'} into \texttt{c}, then meet the character \texttt{'5'} and stop reading for \texttt{c}.\\
        Finally, \texttt{cin} will read \texttt{5} into \texttt{b}.\\
        This is a little tip for you to understand how \texttt{cin} works.
    \end{frame}

    \section{Force data type casting}
    \begin{frame}
        \frametitle{Idea of force data type casting}
        In C++, we should declare a variable with data type before we utlize it. But what if we assign a value with different data type to it?\\
        In this case, C++ will automatically convert the assigned value to the data type of the variable.\\
        For instance, we declare an integer variable \texttt{a}, but we assign a double value \texttt{3.14} to it.\\
        In this case, C++ will automatically convert \texttt{3.14} to \texttt{3} and assign it to \texttt{a}.\\
    \end{frame}
    \begin{frame}
        \frametitle{Conversion rules}
        \begin{itemize}
            \item Integer to double: add a decimal point and a zero.\\
            \item Double to integer: remove the decimal point and the digits after it.\\
            \item Character to integer: convert the character to its ASCII code.\\
            \item Integer to character: convert the ASCII code to its corresponding character.
        \end{itemize}
    \end{frame}

    \section{Selection Statements}
    \begin{frame}
        \frametitle{Preface: How to convert score to GPA?}
        Following is the GPA conversion table of an university.
        \begin{table}
            \centering
            \begin{tabular}{ccl}
                \toprule
                Score & Grade & GPA \\
                \midrule
                80-100 & A & 4.0 \\
                70-79 & B & 2.8 \\
                60-69 & C & 1.5 \\
                0-59 & F & 0.0 \\
                \bottomrule
            \end{tabular}
            \caption{Grade Conversion Table}
        \end{table}
    \end{frame}
    \begin{frame}[fragile]
        \frametitle{Solution without selection statements}
        With only variable operations, we can do it like this:
        \begin{minted}{cpp}
        #include<bits/stdc++.h>
        using namespace std;

        int main(){
            int score;
            double gpa;
            cin >>score;
            gpa=(score>59)*1.5+(score>69)*1.3+(score>79)*1.2;
            cout<<gpa<<'\n';
        }            
        \end{minted}
        
    \end{frame}


\end{document}