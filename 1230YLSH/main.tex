\documentclass[xcolor=dvipsnames]{beamer}

% ==== 主題 ====
\usetheme{metropolis}
\usefonttheme{professionalfonts}          % 不覆蓋你自訂的字型

% ==== 字型 ====
\usepackage{fontspec}
\usepackage{xeCJK}
\renewcommand{\familydefault}{\rmdefault} % 使用 serif 字體(重點)

% 西文字型:Times New Roman 的開源替代品
\setmainfont{TeX Gyre Termes}[
  Ligatures=TeX,
  BoldFont={* Bold},
  ItalicFont={* Italic}
]

% 中文字型(可改為思源宋體、標楷體等)
\setCJKmainfont{Noto Serif CJK TC}
\setCJKsansfont{Noto Sans CJK TC} % 有需要再用
\setCJKmonofont{Noto Sans Mono CJK TC}

% ==== 數學字型(與正文字體一致)====
\usepackage{unicode-math}
\setmathfont{TeX Gyre Termes Math}

% ==== 套件 ====
\usepackage{amsmath, amssymb}
\usepackage{graphicx}
\usepackage{hyperref}
\usepackage{minted}
\usepackage{fvextra}
\usepackage{xcolor}
\usepackage{booktabs}

% ==== 顏色設定(可選)====
\definecolor{MyBlue}{RGB}{3, 55, 105}
\setbeamercolor{structure}{fg=MyBlue}
\setbeamercolor{block title}{bg=MyBlue,fg=white}
\setbeamercolor{block body}{bg=blue!5}
\setbeamertemplate{section in toc}{%
  \inserttocsectionnumber.~\inserttocsection\par
}

\setminted{
    linenos,                % 行號
    frame=lines,            % 上下框線
    framesep=5pt,           % 程式碼與邊框距離
    numbersep=8pt,          % 行號與程式碼距離
    fontsize=\scriptsize,   % 字體大小
    breaklines,             % 自動換行
    tabsize=4,              % tab 寬度
    rulecolor=\color{black},% 框線顏色
    xleftmargin=1.5em       % 左側縮排
}

\title{How should we face this new era}
\author{Tai, Wei Hsuan}
\date{December 30, 2025}
\AtBeginSubsection{
  \begin{frame}{Outline}
    \tableofcontents[currentsection, currentsubsection]
  \end{frame}
}

\begin{document}
\begin{frame}
    \titlepage
\end{frame}

\begin{frame}{Outline}
    \tableofcontents
\end{frame}

\section{About Me}
\begin{frame}
    \frametitle{About Me}
    \begin{itemize}
        \item 戴偉璿
        \item Graduated from YLSH at 112 (YLSH student id: \textbf{9}10466)
        \item Biomedical Engineering, National Taiwan University
        \item CTBC brothers, Philadelphia Phillies, Manchester City
        \item Brawl Stars, Core Keeper, Rim World
        \item Old songs
    \end{itemize}
\end{frame}

\begin{frame}
    \frametitle{What I'm skilled at}
    \begin{itemize}
        \item C++ programming
        \item Datastructures and Algorithms
        \item Machine Learning and Deep Learning
        \item Linear Algebra and Statistics
    \end{itemize}
\end{frame}

\section{About Biomedical Engineering}
\begin{frame}
    \frametitle{What I'm doing now?}
    Employing deep learning methods to generate synthetic CT images.
    \begin{figure}
        \centering
        \includegraphics[width=0.25\linewidth]{src/result_000.png}
        \hspace{0.5em}
        $\rightarrow$
        \hspace{0.5em}
        \includegraphics[width=0.25\linewidth]{src/result_010.png}
        \hspace{0.5em}
        $\rightarrow$
        \hspace{0.5em}
        \includegraphics[width=0.25\linewidth]{src/result_199.png}
        \caption{CT image generation results across training epochs.}
    \end{figure}
\end{frame}
\begin{frame}
    \begin{figure}
        \centering
        \includegraphics[width=0.45\linewidth]{src/result_199.png}
        \hspace{0.5em}
        \includegraphics[width=0.45\linewidth]{src/real.png}
        \caption{Generated CT image (left) vs. Real CT image (right).}
    \end{figure}
\end{frame}

\begin{frame}
    \frametitle{Brief Intro to Biomedical Engineering}
    \begin{itemize}
        \item Utilizing engineering principles to solve medical problems.
        \item Four main areas:
        \begin{itemize}
            \item 生醫材料
            \item 生物力學
            \item 生醫電子
            \item 生醫資訊
        \end{itemize}
        \item Learning basic medicine, biology, chemistry, physics, and engineering at Grade 1 and 2.
        \item Specializing in one of the four areas at Grade 3 and 4.
    \end{itemize}
\end{frame}
{
\usebackgroundtemplate{
    \includegraphics[
        width=\paperwidth,
        height=\paperheight
    ]{src/course_map.pdf}
}
\begin{frame}[plain]
\end{frame}
}


\section{About AI}

\section{About You}

\end{document}
