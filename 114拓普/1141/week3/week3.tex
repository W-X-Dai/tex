\documentclass[xcolor=dvipsnames]{beamer}

% ==== 主題 ====
\usetheme{metropolis}
\usefonttheme{professionalfonts}          % 不覆蓋你自訂的字型

% ==== 字型 ====
\usepackage{fontspec}
\usepackage{xeCJK}
\renewcommand{\familydefault}{\rmdefault} % 使用 serif 字體(重點)

% 西文字型:Times New Roman 的開源替代品
\setmainfont{TeX Gyre Termes}[
  Ligatures=TeX,
  BoldFont={* Bold},
  ItalicFont={* Italic}
]

% 中文字型(可改為思源宋體、標楷體等)
\setCJKmainfont{Noto Serif CJK TC}
\setCJKsansfont{Noto Sans CJK TC} % 有需要再用
\setCJKmonofont{Noto Sans Mono CJK TC}

% ==== 數學字型(與正文字體一致)====
\usepackage{unicode-math}
\setmathfont{TeX Gyre Termes Math}

% ==== 套件 ====
\usepackage{amsmath, amssymb}
\usepackage{graphicx}
\usepackage{hyperref}
\usepackage{minted}
\usepackage{fvextra}
\usepackage{xcolor}
\usepackage{booktabs}

% ==== 顏色設定(可選)====
\definecolor{MyBlue}{RGB}{3, 55, 105}
\setbeamercolor{structure}{fg=MyBlue}
\setbeamercolor{block title}{bg=MyBlue,fg=white}
\setbeamercolor{block body}{bg=blue!5}
\setbeamertemplate{section in toc}{%
  \inserttocsectionnumber.~\inserttocsection\par
}

\setminted{
    linenos,                % 行號
    frame=lines,            % 上下框線
    framesep=5pt,           % 程式碼與邊框距離
    numbersep=8pt,          % 行號與程式碼距離
    fontsize=\scriptsize,   % 字體大小
    breaklines,             % 自動換行
    tabsize=4,              % tab 寬度
    rulecolor=\color{black},% 框線顏色
    xleftmargin=1.5em       % 左側縮排
}

\title{Selection Statements \& Logical Operators}
\author{Tai, Wei Hsuan}
\date{week 3}

\begin{document}
	\begin{frame}
		\titlepage
	\end{frame}


    % course material qrcode
    \begin{frame}
        \frametitle{\href{https://drive.google.com/drive/folders/14Tkn-rddw0k1obeOxkWi00S43M0e9wlW?usp=sharing}{Course Materials}}
        \begin{figure}
            \centering
            \includegraphics[width=0.7\textwidth]{src/qrcode.png}
        \end{figure}
    \end{frame}

    \begin{frame}
        \frametitle{Outline}
        \tableofcontents
    \end{frame}

    \section{Recap}
    \begin{frame}
        \frametitle{Remember}
        \begin{itemize}
            \item Every statement ends with a semicolon \textbf{(;)}.
            \item Programs always start from the \alert{main()} function.
            \item All functional code needs to be inside a function.
            \item All functions must be enclosed in curly braces \textbf{\{\}}.
        \end{itemize}
    \end{frame}
    \begin{frame}[fragile]
        \frametitle{Basic structure of C++}
        \begin{minted}{cpp}
            #include <bits/stdc++.h>
            using namespace std;

            int main() {
                // code here
                return 0;
            }
        \end{minted}
    \end{frame}
    \begin{frame}[fragile]
        \frametitle{Variables Declaration}
        \begin{minted}{cpp}
            int a;
            double b, pi=3.14;
            char c='A';
        \end{minted}
    \end{frame}
    \begin{frame}[fragile]
        \frametitle{Basic Input/Output}
        \begin{minted}{cpp}
            int a, b, c;
            cin >>a>>b>>c;
            cout<<"The result of a*10 is"<<a*10<<endl;
        \end{minted}
    \end{frame}
    \begin{frame}[fragile]
        \frametitle{What's the value of each variable?}
        \begin{minted}{cpp}
            int a, b;
            char c;
            //input: 10+5
            cin >>a>>c>>b;
        \end{minted}
    \end{frame}
    \begin{frame}
        \frametitle{A little tip}
        The result is \texttt{a=10, b=5, c='+'}\\
        In C++, the input stream will read the whole input until it meets a whitespace, newline or another data type.\\
        Therefore, \texttt{cin} will read \texttt{10} into \texttt{a}, then meet the character \texttt{'+'} and stop reading for \texttt{a}.\\
        Next, \texttt{cin} will read \texttt{'+'} into \texttt{c}, then meet the character \texttt{'5'} and stop reading for \texttt{c}.\\
        Finally, \texttt{cin} will read \texttt{5} into \texttt{b}.\\
        This is a little tip for you to understand how \texttt{cin} works.
    \end{frame}

    \section{Force data type casting}
    \begin{frame}
        \frametitle{Idea of force data type casting}
        In C++, we should declare a variable with data type before we utlize it. But what if we assign a value with different data type to it?\\
        In this case, C++ will automatically convert the assigned value to the data type of the variable.\\
        For instance, we declare an integer variable \texttt{a}, but we assign a double value \texttt{3.14} to it.\\
        In this case, C++ will automatically convert \texttt{3.14} to \texttt{3} and assign it to \texttt{a}.\\
    \end{frame}
    \begin{frame}
        \frametitle{Conversion rules}
        \begin{itemize}
            \item Integer to double: add a decimal point and a zero.\\
            \item Double to integer: remove the decimal point and the digits after it.\\
            \item Character to integer: convert the character to its ASCII code.\\
            \item Integer to character: convert the ASCII code to its corresponding character.
        \end{itemize}
    \end{frame}
    \begin{frame}
        \frametitle{Little Practice}
        We give all letters a score according to their position in the alphabet.\\
        For example, \texttt{a=1, b=2, c=3, ..., z=26}.\\
        Given a lowercase letter, tell me its score.\\
        Hint: you may need to use force data type casting.
    \end{frame}
    \begin{frame}[fragile]
        \frametitle{Solution}
        We know that all letters have their own ASCII code, and the ASCII code is continuous.\\
        Therefore, we can just calculate the distance between the given letter and \texttt{'a'}.\\
        Remember to add \texttt{1} to the result, because \texttt{'a'} is the first letter.\\
        \begin{minted}{cpp}
            char c;
            cin >>c;
            cout<<(int)(c-'a'+1)<<'\n';
        \end{minted}
        You can add brackets to fix the target data type.
    \end{frame}

    \section{Selection Statements}
    \begin{frame}
        \frametitle{Preface: How to convert score to GPA?}
        Following is the GPA conversion table of an university.
        \begin{table}
            \centering
            \begin{tabular}{ccl}
                \toprule
                Score & Grade & GPA \\
                \midrule
                80-100 & A & 4.0 \\
                70-79 & B & 2.8 \\
                60-69 & C & 1.5 \\
                0-59 & F & 0.0 \\
                \bottomrule
            \end{tabular}
            \caption{Grade Conversion Table}
        \end{table}
    \end{frame}
    \begin{frame}[fragile]
        \frametitle{Solution without selection statements}
        With only variable operations, we can do it like this:
        \begin{minted}{cpp}
        #include<bits/stdc++.h>
        using namespace std;

        int main(){
            int score;
            double gpa;
            cin >>score;
            gpa=(score>59)*1.5+(score>69)*1.3+(score>79)*1.2;
            cout<<gpa<<'\n';
        }            
        \end{minted}
        But this notation is not easy to read. We have a more readable way to do it.
    \end{frame}
    \begin{frame}
        \frametitle{Overview of Selection Statements}
        Selection statements are used to select a block of code to be executed based on a condition.\\
        In C++, we have three selection statements: \texttt{if}, \texttt{if-else} and \texttt{switch}.\\
        In fact, \texttt{switch} is not commonly used, so we will focus on \texttt{if} and \texttt{if-else} today.
    \end{frame}
    \begin{frame}[fragile]
        \frametitle{Structure of Selection Statements}
        \begin{minted}{cpp}
            if(condition) {
                // code to be executed if condition is true
            }else if(condition2){
                // code to be executed if condition2 is true
            }else{
                // code to be executed if both condition and condition2 are false
            }
        \end{minted}
        Note that condition must be a boolean expression, which means it can only be \texttt{true} or \texttt{false}.
        In C++, any non-zero value is considered as \texttt{true}, and \texttt{0} is considered as \texttt{false}.\\
        We often use a logical expression as the condition.
    \end{frame}
    \begin{frame}[fragile]
        \frametitle{Back to GPA conversion}
        \begin{minted}{cpp}
        #include<bits/stdc++.h>
        using namespace std;

        int main(){
            int score;
            double gpa=0;
            cin >>score;
            if(score>79){
                gpa=4.0;
            }else if(score>69){
                gpa=2.8;
            }else if(score>59){
                gpa=1.5;
            }
            cout<<gpa<<'\n';
        }            
        \end{minted}
    \end{frame}
    \begin{frame}
        \frametitle{Practice}
        Given a letter, tell me it is uppercase or lowercase.\\
        Hint: Remeber that letters are continuous in ASCII table.
    \end{frame}
    \begin{frame}[fragile]
        \frametitle{Solution}
        \begin{minted}{cpp}
        #include<bits/stdc++.h>
        using namespace std;

        int main(){
            char c;
            cin >> c;
            if(c>='A' && c<='Z'){
                cout<<"Uppercase"<<endl;
            }else if(c>='a' && c<='z'){
                cout<<"Lowercase"<<endl;
            }
        }
        \end{minted}
    \end{frame}

    \section{Logical Operators}
    \begin{frame}
        \frametitle{Overview of Logical Operators}
        In Selection Statements, logical expressions are often used as the condition.\\
        In fact, computers would calculate the logical expressions to determine whether the condition is true or false, then return the result to the selection statements.\\
    \end{frame}
    \begin{frame}
        \frametitle{Logical Operators}
        In C++, we have three logical operators: \texttt{AND (\&\&)}, \texttt{OR (||)} and \texttt{NOT (!)}, and we've learned them in the previous course. Let's take a recap.
        \begin{table}
            \centering
            \begin{tabular}{cc|c|c|c}
                \toprule
                A & B & A \texttt{\&\&} B & A \texttt{||} B & !A \\
                \midrule
                0 & 0 & 0 & 0 & 1 \\
                0 & 1 & 0 & 1 & 1 \\
                1 & 0 & 0 & 1 & 0 \\
                1 & 1 & 1 & 1 & 0 \\
                \bottomrule
            \end{tabular}
            \caption{Truth Table of Logical Operators}
        \end{table}
    \end{frame}
    \begin{frame}[fragile]
        \frametitle{Usage}
        With logical operators, we can combine multiple conditions into one condition.\\
        For example, we want to check if a number is between 1 and 10.\\
        We can do it like this:
        \begin{minted}{cpp}
            int a;
            cin >>a;
            if(a>=1 && a<=10){
                cout<<"a is between 1 and 10"<<endl;
            }
        \end{minted}
    \end{frame}
    \begin{frame}
        \frametitle{Practice}
        Given two integers, tell me how many even numbers are between them (inclusive).
        For example, given \texttt{3} and \texttt{9}, the even numbers between them are \texttt{4, 6, 8}.\\
        In this problem, loop is forbidden because the number of integers between the two given integers may be too large.(e.g. $10^{15}$)
    \end{frame}
    \begin{frame}
        \frametitle{Practice}
        There are 3 ants on a stick of length $L$.
        The positions of the ants are $a$, $b$ and $c$.
        Each ant will randomly move to the left or right with equal probability.
        When two ants meet, they will turn back and move in the opposite direction.
        When an ant reaches the end of the stick, it will fall off.
        Calculate the expected time for all ants to fall off the stick.
    \end{frame}
    \begin{frame}
        \frametitle{Practice}
        There's a robot on a 2D plane.
        The robot starts at the origin \texttt{(0, 0)} and faces the positive x-axis.
        At $t^{th}$ second, the robot will move forward $t$ units, then turn left 90 degrees.
        After $n$ seconds, where will the robot be?
    \end{frame}
\end{document}