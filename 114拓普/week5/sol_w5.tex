\documentclass[12pt,a4paper]{article}
\usepackage[margin=2cm]{geometry}
\usepackage{xeCJK}
\usepackage{fontspec}
\setCJKmainfont{Noto Serif CJK TC}[Script=CJK]
\usepackage{amsmath,amssymb}
\usepackage{graphicx}
\usepackage{fancyhdr}
\setlength{\headheight}{14.5pt}
\addtolength{\topmargin}{-2.5pt}
\usepackage{hyperref}
\usepackage{listings}
\usepackage{enumitem}
\usepackage{titlesec}
\usepackage{caption}
\usepackage{indentfirst}
\usepackage{booktabs}
\usepackage{longtable}
\usepackage{multirow}
\usepackage{array}
\usepackage{tabularx}
\usepackage{float}
\usepackage{minted}
\setlength{\parindent}{2em}
\pagestyle{fancy}
\fancyhf{}
\cfoot{\thepage}
\linespread{1.3}
\setminted{
    linenos,                % 行號
    frame=lines,            % 上下框線
    framesep=5pt,           % 程式碼與邊框距離
    numbersep=8pt,          % 行號與程式碼距離
    fontsize=\scriptsize,   % 字體大小
    breaklines,             % 自動換行
    tabsize=4,              % tab 寬度
    rulecolor=\color{black},% 框線顏色
    xleftmargin=1.5em       % 左側縮排
}

\title{Week 5 Reference Solutions}
\author{Tai, Wei-Hsuan}
\date{\today}

\begin{document}

\maketitle

\lhead{Week 5 Solutions}
\rhead{\today}

\begin{enumerate}
    \item Given n numbers, list them in reverse order.
    \begin{itemize}
        \item Idea: Just output the array from the last element to the first element.
        \item Sample Input:
        \begin{minted}{text}
5
10 20 30 40 50
        \end{minted}
        \item Sample Output:
        \begin{minted}{text}
50 40 30 20 10
        \end{minted}
        \item Solution:
        \begin{minted}{cpp}
#include<bits/stdc++.h>
using namespace std;

int main(){
    int arr[10005], n;
    cin >>n;
    for(int i=0;i<n;++i){
        cin >>arr[i];
    }
    for(int i=n-1;i>=0;--i){
        cout<<arr[i]<<' ';
    }
}
        \end{minted}
    \end{itemize}
    \item Given n numbers, list the frequency of each number. Note that the numbers are all between 0 and 100.
    \begin{itemize}
        \item Idea: Use an array to count the frequency of each number.
        \item Sample Input:
        \begin{minted}{text}
7
1 1 2 2 3 3 3 3
        \end{minted}
        \item Sample Output:
        \begin{minted}{text}
1:2
2:2
3:3
        \end{minted}
        \item Solution:
        \begin{minted}{c++}
#include<bits/stdc++.h>
using namespace std;

int main(){
    int arr[105]={0}, n, mx=-1;
    cin >>n;
    for(int i=0;i<n;++i){
        int tmp;
        cin >>tmp;
        mx=max(mx, tmp);
        ++arr[tmp];
    }   
    for(int i=0;i<=mx;++i){
        if(arr[i])cout<<i<<":"<<arr[i]<<'\n';
    }
}
        \end{minted}
    \end{itemize}
    \item The formula of standard deviation is:$SD=\sqrt{\frac{1}{n}\sum_{i=1}^{n}(x_i-\mu)^2}$, where $\mu$ is the mean of the n numbers. Given n numbers, calculate the standard deviation.
    \begin{itemize}
        \item Idea: First calculate the mean, then calculate the standard deviation using the formula.
        \item Sample Input:
        \begin{minted}{text}
5
1 2 3 4 5
        \end{minted}
        \item Sample Output:
        \begin{minted}{text}
1.41421
        \end{minted}
        \item Solution:
        \begin{minted}{c++}
#include<bits/stdc++.h>
using namespace std;

int main(){
    int arr[10005], mean=0, n;
    
    cin >>n;
    for(int i=0;i<n;++i){
        cin >>arr[i];
        mean+=arr[i];
    }
    mean/=n;

    double sd=0;
    for(int i=0;i<n;++i){
        sd+=(arr[i]-mean)*(arr[i]-mean);
    }
    sd=sqrt(sd/n);

    cout<<sd;
}
        \end{minted}
    \end{itemize}
    \item Given n numbers, find the length of the longest continuous increasing subsequence.
    \begin{itemize}
        \item Idea: Enumerate the start of the subsequence, and extend it as long as possible.
        \item Sample Input:
        \begin{minted}{text}
13
10 22 9 33 21 50 41 60 80 3 5 7 8
        \end{minted}
        \item Sample Output:
        \begin{minted}{text}
4
        \end{minted}
        \item Solution:
        \begin{minted}{c++}
#include<bits/stdc++.h>
using namespace std;

int main(){
    int arr[10005], n, mx=-1;
    cin >>n;
    for(int i=0;i<n;++i)cin >>arr[i];
    
    for(int i=0;i<n;++i){
        for(int j=0;i+j<n;++j){
            int now=i+j;
            if(j==0 || arr[now]>arr[now-1]){
                mx=max(mx, j+1);
            }
            else break;
        }
    }
    cout<<mx<<'\n';
}
        \end{minted}
    \end{itemize}
    \item Given n numbers, find their greatest common divisor (GCD) and least common multiple (LCM).
    \begin{itemize}
        \item Idea: Enumerate all numbers to find the GCD and LCM.
        \item Sample Input:
        \begin{minted}{text}
5
12 18 24 30 42
        \end{minted}
        \item Sample Output:
        \begin{minted}{text}
6 2520
        \end{minted}
        \item Solution:
        \begin{minted}{c++}
#include<bits/stdc++.h>
using namespace std;

int main(){
    int arr[10005], n, GCD, LCM, mi=1e9, ma=-1e9;
    cin >>n;
    for(int i=0;i<n;++i)cin >>arr[i], mi=min(mi, arr[i]), ma=max(ma, arr[i]);

    GCD=1;
    for(int i=1;i<=mi;++i){
        bool flag=true;
        for(int j=0;j<n;++j){
            if(arr[j]%i!=0){
                flag=false;
                break;
            }
        }
        if(flag)GCD=i;
    }

    LCM=ma;
    for(int i=ma;;++i){
        bool flag=true;
        for(int j=0;j<n;++j){
            if(i%arr[j]!=0){
                flag=false;
                break;
            }
        }
        if(flag){
            LCM=i;
            break;
        }
    }
    cout<<GCD<<" "<<LCM<<'\n';
}
        \end{minted}
    \end{itemize}
    \item Given n numbers, find the maximum sum of any contiguous subarray (elements in the array may be negative).
    \begin{itemize}
        \item Idea: Enumerate the start and end of the subarray, and calculate the sum.
        \item Sample Input:
        \begin{minted}{text}
7
-2 1 -3 4 1 -1 5
        \end{minted}
        \item Sample Output:
        \begin{minted}{text}
9
        \end{minted}
        \item Solution:
        \begin{minted}{c++}
#include<bits/stdc++.h>
using namespace std;

int main(){
    int arr[10005], n;
    cin >>n;
    for(int i=0;i<n;++i)cin >>arr[i];

    int ma=-1e9, cur=0;
    for(int i=0;i<n;++i){
        for(int j=i;j<n;++j){
            cur+=arr[j];
            ma=max(ma, cur);
        }
        cur=0;
    }

    cout<<ma<<'\n';
}
        \end{minted}
    \end{itemize}
\end{enumerate}

\end{document}