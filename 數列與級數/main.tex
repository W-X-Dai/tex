\documentclass[12pt]{article}
\usepackage{xeCJK}
\usepackage{fontspec}
\setCJKmainfont{Noto Serif CJK TC}[Script=CJK]
\usepackage{amsmath, amssymb, amsthm}
\usepackage{geometry}
\usepackage{tcolorbox}
\usepackage{fancyhdr}
\geometry{a4paper, margin=2.5cm}

\lhead{數列與級數的極限}
\rhead{數列}

\begin{document}


\begin{tcolorbox}[colframe=black!60!blue, colback=blue!5!white, arc=4pt, boxrule=1pt]
\section{數列}
\begin{enumerate}
    \item \textbf{數列的定義}:數列是由一系列數字按照一定的順序排列而成的集合。每個數字稱為數列的項,通常用符號 $a_n$ 表示第 $n$ 項。
    \item \textbf{數列的表示法}:數列可以用括號表示,如 $(a_1, a_2, a_3, \ldots)$,
    或用尖括號表示,如 $<a_n>_{n=1}^{\infty}$。
    \item \textbf{數列的類型}:\textbf{有限數列}和\textbf{無窮數列}。有限數列是指包含有限個數字的數列,而無限數列則包含無窮多個數字。
    \item \textbf{數列的第$n$項}:表示數列的第 $n$ 項,
    除了可以直接寫出$a_n$的公式(如:$a_n = n^2$),
    還可以用遞迴公式來定義,如:$a_1 = 1, a_{n+1} = a_n + 2$。

\end{enumerate}

\end{tcolorbox}

例題:
\begin{enumerate}
    \item 數列A:$1, \frac{1}{2}, \frac{1}{3}, \ldots$,請寫出這個數列的第 $n$ 項公式。\vspace{2cm}
    \item 已知一個數列 $a_n$ 的遞迴公式為 $a_1 = 2, a_{n+1} = 3a_n + 1$,請求出他的一般式。\vspace{2cm}
    \item 假設有一個數列$<a_n>$的前$n$項和為$S_n = 2n^2 + 3n$,請求出$a_n$的公式。\vspace{2cm}
\end{enumerate}
\newpage

\begin{tcolorbox}[colframe=black!60!blue, colback=blue!5!white, arc=4pt, boxrule=1pt]
\section{極限}
\begin{enumerate}
    \item \textbf{數列的極限}:如果說$n$在趨近於無限大的時候,數列 $a_n$ 
    的值趨近於某個固定的數 $L$,則稱 $L$ 為數列 $a_n$ 的極限,
    記作 $\displaystyle\lim_{n \to \infty} a_n = L$,或是 $a_n \to L$。
    \item \textbf{發散與收斂}:如果數列的極限存在,則稱該數列為\textbf{收斂數列};
    如果極限不存在,則稱為\textbf{發散數列}。
    如果計算結果為$\frac{0}{0}$,則需要進一步化簡整理。
    \item \textbf{極限的四則運算}:
    
    假設數列 $a_n \to A$ 和 $b_n \to B$,則:
    \begin{itemize}
        \item $\displaystyle\lim_{n \to \infty} (a_n + b_n) = A + B$
        \item $\displaystyle\lim_{n \to \infty} (a_n - b_n) = A - B$
        \item $\displaystyle\lim_{n \to \infty} (a_n b_n) = AB$
        \item 如果 $B \neq 0$,則 $\displaystyle\lim_{n \to \infty} \left(\frac{a_n}{b_n}\right) = \frac{A}{B}$
    \end{itemize}
    \item \textbf{夾擠定理}:如果有數列 $a_n \leq b_n \leq c_n$,且 $\displaystyle\lim_{n \to \infty} a_n = L$ 和 $\displaystyle\lim_{n \to \infty} c_n = L$,
    則 $\displaystyle\lim_{n \to \infty} b_n = L$。
\end{enumerate}

\end{tcolorbox}

例題:
\begin{enumerate}
    \item 判斷以下數列是否收斂,如果是則求出極限:
    \begin{enumerate}
        \item $a_n = \frac{1}{n}$\vspace{0.7cm}
        \item $b_n = \frac{n^2 + 1}{n^2 - 1}$\vspace{0.7cm}
        \item $c_n = 1, -1, 1, -1, \ldots$\vspace{0.7cm}
        \item $d_n = \left(\frac{-1}{2}\right)^n$\vspace{0.7cm}
        \item $e_n = \frac{n^2 + 2n + 1}{3n + 2}$\vspace{0.7cm}
        \item $f_n = \frac{3n+2}{n^2 + 2n + 1}$\vspace{0.7cm}
        \item $g_n = \left(2-\frac{1}{n}\right)\left(3+\frac{3}{n^2}\right)$\vspace{0.7cm}
        \item $h_n = \frac{n^2}{2n-1}-\frac{n^2}{2n+1}$\vspace{0.7cm}
    \end{enumerate}
    \item 數列$<a_n>$中,$a_1=0, a_{n+1}=a_n+2n-1, n> 1$,求$a_{100}$\vspace{2cm}
    \item 若數列$<a_n>$的前$k$項和為$S_k = 2^{k+1}(k^2-2k)$,請求出$a_{10}$。\vspace{2cm}
    \item 已知$A, B$皆為無窮數列,請選出正確的選項:
    \begin{enumerate}
        \item 若$\displaystyle\lim_{n\to \infty}a_n=A$,則$\displaystyle\lim_{n\to \infty}a_{n+1}=A$
        \item 若$\displaystyle\lim_{n\to \infty}a_n=A$,則$\displaystyle\lim_{n\to \infty}a_{2n}=A$
        \item 若$\displaystyle\lim_{n\to \infty}a_{2n}=A$,則$\displaystyle\lim_{n\to \infty}a_{n}=A$
        \item 若$\displaystyle\lim_{n\to \infty}a_{2n}=\displaystyle\lim_{n\to \infty}a_{2n+1}=A$,則$\displaystyle\lim_{n\to \infty}a_{n+1}=A$
        \item 若$\displaystyle\lim_{n\to \infty}a_n=A$,則$\displaystyle\lim_{n\to \infty}\sqrt a_{n}=\sqrt A$
    \end{enumerate}
    \item 請計算$\displaystyle\lim_{n\to\infty}\frac{2^{n-1}+5\cdot 3^{n+1}-6\cdot 4^{n-1}}{3\cdot 2^{n+1}-4\cdot 3^{n-1}+7\cdot 4^{n+1}}$。\vspace{2cm}

    \item 設$<a_n>$收斂,且$\displaystyle\lim_{n\to \infty}\frac{2^n+(-3)^n\cdot a_n}{2^n-{-3}^n}=\frac{2}{3}$,請計算$\displaystyle\lim_{n\to\infty}a_n$
    \vspace{2cm}
    \item 若多項式$x^n-1$除以$x-\frac{1}{4}$的商為$Q_n(x)$,餘式為$r_n$,求$\displaystyle\lim_{n\to\infty}r_n$和
    $\displaystyle\lim_{n\to\infty}Q_n(1)$。\vspace{2cm}
    \item 座標平面上,函數$y=2^{-x}$與$y=\cos(2x+\pi)+\frac{1}{2}$的圖形在$y$軸右側的交點由左而右依序為$A_1, A_2, A_3...$,
    若以$x_k$表示$A_k$的座標,定義數列$<c_n>=<x_{2n}-x_{2n-1}>$,求$\displaystyle\lim_{n\to\infty}c_n$\vspace{2cm}
\end{enumerate}

\end{document}
