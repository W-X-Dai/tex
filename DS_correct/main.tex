\documentclass[12pt,a4paper]{article}
\usepackage[margin=2cm]{geometry}
\usepackage{xeCJK}
\usepackage{fontspec}
\setCJKmainfont{Noto Serif CJK TC}[Script=CJK]
\usepackage{amsmath,amssymb}
\usepackage{graphicx}
\usepackage{fancyhdr}
\setlength{\headheight}{14.5pt}
\addtolength{\topmargin}{-2.5pt}
\usepackage{hyperref}
\usepackage{listings}
\usepackage{enumitem}
\usepackage{titlesec}
\usepackage{caption}
\usepackage{indentfirst}
\setlength{\parindent}{2em}
\pagestyle{fancy}
\usepackage{array}
\fancyhf{}
\cfoot{\thepage}
\linespread{1.3}

\usepackage{xcolor}
\lstset{
    basicstyle=\ttfamily\footnotesize,  % 字型與大小
    keywordstyle=\color{blue},
    commentstyle=\color{gray},
    stringstyle=\color{orange},
    numbers=left,                       % 行號在左側
    numberstyle=\tiny\color{gray},
    stepnumber=1,                       % 每行都顯示行號
    numbersep=5pt,
    backgroundcolor=\color{white},
    frame=single,                       % 加上框線
    breaklines=true,                    % 長行自動換行
    tabsize=2,
    language=C++                     % 可以換成 \texttt{C++}, \texttt{Java}, etc.
}

\title{資料結構其中考訂正}
\author{B12508026戴偉璿}
\date{\today}

\begin{document}

\maketitle

\lhead{資料結構期中考訂正}
\rhead{B12508026戴偉璿}
\newpage

\section*{Problem A}

\subsection*{A1.解釋Shunting-yard演算法}
\begin{enumerate}
    \item 前置處理:準備一個輸出的隊列以及一個暫存用的\texttt{stack}
    \item 讀入下個\texttt{token}
    \item 
    \begin{enumerate}
        \item 如果這個\texttt{token} 是個運算符號,判斷他是「左結合」還是「右結合」(當運算子優先順序相同應該先處理哪邊的)
        \item 根據結合性與優先順序決定是否要把\texttt{stack}中的運算子丟到出隊列:
        \begin{enumerate}
            \item 左結合:如果\texttt{stack}頂端的運算子優先順序大於等於當前\texttt{token}的優先順序,不斷從\texttt{stack}中\texttt{pop}出來
            並丟到輸出隊列
            \item 右結合:如果\texttt{stack}頂端的運算子優先順序大於當前\texttt{token}的優先順序時,不斷從\texttt{stack}彈出並送入輸出隊列
        \end{enumerate}
        \item 處理完後,將目前\texttt{token}丟到\texttt{stack}中,如果是括號則掠過
    \end{enumerate}
    \item 如此往復直到所有\texttt{token}都讀入,最後把\texttt{stack}中剩下的元素全部取出,丟到輸出隊列中,如果是括號則略過
\end{enumerate}
\newpage
\subsection*{A2:處理$(a+b)\times(c-d/e)+f\times(g-h)$}

觀察這個方程式,裡面每個運算子都是「左結合」(沒有出現次方符號),以下是逐步操作的表格
\vspace{0.2cm}

\newcommand{\tok}[1]{\texttt{#1}}

\begin{tabular}{|c|>{\raggedright}p{5cm}|>{\raggedright\arraybackslash}p{6cm}|}
\hline
\texttt{token} & \texttt{stack}的狀態 & 輸出隊列 \\
\hline
\tok{(}   & (                         &                              \\
\tok{a}   & (                         & $a$                          \\
\tok{+}   & ( +                       & $a$                          \\
\tok{b}   & ( +                       & $a b$                   \\
\tok{)}   &                           & $a b +$            \\
\tok{*}   & *                         & $a b +$            \\
\tok{(}   & * (                       & $a b +$            \\
\tok{c}   & * (                       & $a b + c$     \\
\tok{-}   & * ( -                     & $a b + c$     \\
\tok{d}   & * ( -                     & $a b + c d$ \\
\tok{/}   & * ( - /                   & $a b + c d$ \\
\tok{e}   & * ( - /                   & $a b + c d e$ \\
\tok{)}   & *                         & $a b + c d e / -$ \\
\tok{+}   & +                         & $a b + c d e / - *$ \\
\tok{f}   & +                         & $a b + c d e / - * f$ \\
\tok{*}   & + *                       & $a b + c d e / - * f$ \\
\tok{(}   & + * (                     & $a b + c d e / - * f$ \\
\tok{g}   & + * (                     & $a b + c d e / - * f g$ \\
\tok{/}   & + * ( /                   & $a b + c d e / - * f g$ \\
\tok{h}   & + * ( /                   & $a b + c d e / - * f g h$ \\
\tok{)}   & + *                       & $a b + c d e / - * f g h / $ \\
結束 &                        & $a b + c d e / - * f g h / * +$ \\
\hline
\end{tabular}

在結束時是將\texttt{stack}中的元素一個一個取出再丟到輸出隊列中,因此最後丟到輸出隊列的順序是先$*$之後才是$+$


最終答案:
$
a b + c d e / - * f g h / * +
$

\newpage
\subsection*{A3.計算A2.的結果}

運算規則:準備一個紀錄答案的\texttt{stack},從頭開始讀取前序式。
如果當前讀取的\texttt{token}是數字就先存起來,如果當前讀取到的\texttt{token}是運算符號
則從\texttt{stack}中取出最上面的兩個元素進行該運算,結束後再放回\texttt{stack}中。如此往復,直到最後整個\texttt{stack}中
會存在唯一的數字,就是答案。

\begin{center}
\begin{tabular}{|c| p{10cm}|}
\hline
\texttt{token} & \texttt{stack}的內容 \\
\hline
\texttt{a} & $[a]$ \\
\texttt{b} & $[a,\ b]$ \\
\texttt{+} & $[a + b]$(取出$a, b$,進行運算後把$a+b$放回去) \\
\texttt{c} & $[a + b,\ c]$ \\
\texttt{d} & $[a + b,\ c,\ d]$ \\
\texttt{e} & $[a + b,\ c,\ d,\ e]$ \\
\texttt{/} & $[a + b,\ c,\ d / e]$ \\
\texttt{-} & $[a + b,\ c - d / e]$ \\
\texttt{*} & $[(a + b) * (c - d / e)]$ \\
\texttt{f} & $[(a + b)(c - d / e),\ f]$ \\
\texttt{g} & $[(a + b)(c - d / e),\ f,\ g]$ \\
\texttt{h} & $[(a + b)(c - d / e),\ f,\ g,\ h]$ \\
\texttt{/} & $[(a + b)(c - d / e),\ f,\ g / h]$ \\
\texttt{*} & $[(a + b)(c - d / e),\ f * (g / h)]$ \\
\texttt{+} & $[(a + b)(c - d / e) + f (g / h)]$ \\
\hline
\end{tabular}    
\end{center}




\end{document}