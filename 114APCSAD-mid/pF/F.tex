\documentclass[11pt,a4paper]{article}

\usepackage{indentfirst}
\usepackage{amssymb}
\usepackage{subcaption}
\usepackage{graphicx}
\usepackage{longtable}
\usepackage{fancyhdr}
\usepackage{xeCJK}
\usepackage{amsmath}
\usepackage{amssymb}
\usepackage{ulem}
\usepackage{xcolor}
\usepackage{fancyvrb}
\usepackage{listings}
\usepackage{soul}
\usepackage{hyperref}
	\lstdefinestyle{C}{
   		language=C, 
   		basicstyle=\ttfamily\bfseries,
    	numbers=left, 
    	numbersep=5pt,
    	tabsize=4,
    	frame=single,
   	 	commentstyle=\itshape\color{brown},
    	keywordstyle=\bfseries\color{blue},
   	 	deletekeywords={define},
    	morekeywords={NULL,bool}
	}	

\setCJKmainfont{Noto Serif CJK TC}[Script=CJK]
\setCJKmonofont{Noto Sans Mono CJK TC}[Script=CJK]
 
\voffset -20pt
\textwidth 410pt
\textheight 650pt
\oddsidemargin 20pt
\newcommand{\XOR}{\otimes}
\linespread{1.2}\selectfont

\pagestyle{fancy}
\lhead{114APCS進階班期中比賽}

\begin{document}

\begin{center}
\section*{F. Bicycle NTU}
\end{center}

\section*{Description}


台大有大量的腳踏車,但停車空間有限,因此學生常常把其他人的車推擠進狹小空間,導致停車場擁擠。
這樣不僅會讓學生找不到自己的車,也會增加失竊的風險。為了更了解這個問題,總務處決定分析學生的停車行為。

為簡化分析,每個停車格視為一個節點,所有停車格以小徑相連構成一棵樹狀結構。
兩格之間的路徑是唯一的,設 $w(x, y)$ 表示連接 $x$ 與 $y$ 的路徑所需時間,
從 $x$ 到 $z$ 的總花費時間為路徑中所有邊的總和。

每個停車格 $x$ 設有座標範圍 $[1, c_x]$,容量為 $c_x$,可停放一輛或多輛腳踏車。
學生可以自由選擇停放或移動腳踏車,但學校會定期進行整理與清除,有些車會被拖吊到水源校區,學生需搭校車前往領回。

有以下幾種操作:

\begin{enumerate}
\item \textbf{Park} $s, x, p$:學生 $s$ 嘗試將腳踏車停在格子 $x$ 的位置 $p$,規則如下:
\begin{itemize}
\item 若 $p$ 為空,直接停在 $p$。
\item 若 $p$ 被佔用,找出距離 $p$ 最近的空位(優先靠左)。
\item 若整數位置皆已滿,則:
\begin{itemize}
\item 若最左邊的車不在 $p$,插入 $p$ 左側兩車中間(例如 $1, 2, 3$ 被佔,用戶想停 $3$,實際會停在 $5/2$)。
\item 否則插入 $p$ 右側兩車中間(例如停在 $1$ 時,實際會停在 $5/4$)。
\end{itemize}
\end{itemize}

\item \textbf{Move} $s, y, p$:學生 $s$ 將車從原格 $x$ 移至新格 $y$ 的位置 $p$,並計算 $x$ 到 $y$ 的移動時間。若 $x = y$,不移動,耗時為 0。

\item \textbf{Clear} $x, t$:在時間 $t$ 清空格 $x$,將所有車移至水源校區。學生 $s$ 將於 $t + \ell_s$ 被通知來領車。

\item \textbf{Rearrange} $x, t$:在時間 $t$ 清除所有非法停車(非整數座標)之車輛並移至水源校區,學生 $s$ 將於 $t + \ell_s$ 被通知。

\item \textbf{Fetch} $t$:時間 $t$ 校車發車至水源,所有已被通知可領車的學生一同前往,領回後不會立即停車,會等之後再用 Park 操作。

\item \textbf{Rebuild} $x, y, d$(額外):將連接 $x$ 與 $y$ 的邊權重 $w(x, y)$ 改為 $d$,保證 $x, y$ 原本有邊。
\end{enumerate}

\newpage
\section*{Input}

第一行為三整數 $n, q$:停車格數、操作數

第二行有 $n$ 個整數:各停車格容量 $c_x$

接下來 $n-1$ 行,每行三整數 $x, y, w$:表示停車格 $x, y$ 之間有一條距離為 $w$ 的小徑

接下來 $q$ 行操作:格式如下(前綴為操作類型代碼):
\begin{itemize}
\item \textbf{Park}: 0 s x p
\item \textbf{Move}: 1 s y p
\item \textbf{Clear}: 2 x t
\item \textbf{Rearrange}: 3 x t
\item \textbf{Fetch}: 4 t
\item \textbf{Rebuild}: 5 x y d
\end{itemize}

其中 $s$ 為學生編號,$x, y$ 為停車格編號,$p$ 為位置,$t$ 為時間,$d$ 為新距離
\section*{Output}

對於每個操作,輸出格式如下:

\begin{itemize}
\item \textbf{Park}: 輸出 \\
	\texttt{[s] parked at ([x], [fp]).} \\
	,其中 \texttt{fp} 為最簡分數或整數
\item \textbf{Move}: 輸出 \\
	\texttt{[s] moved to [y] in [t] seconds.}
\item \textbf{Clear}: 無輸出
\item \textbf{Rearrange}: 輸出 \\
	\texttt{Rearranged [n] bikes in [x].} \\
	,其中 \texttt{n} 為被移除的腳踏車數量
\item \textbf{Fetch}: 輸出 \\ 
	\texttt{At [t], [n] bikes was fetched.} \\
	,其中 \texttt{n} 為被領回的腳踏車數量
\item \textbf{Rebuild}: 無輸出
\end{itemize}

\newpage
\section*{Sample 1}
\begin{longtable}[!h]{|p{0.5\textwidth}|p{0.5\textwidth}|}
\hline
\textbf{Input} & \textbf{Output} \\
\hline
\parbox[t]{0.5\textwidth}{\ttfamily
1 5 3\\
3\\
0 0 0 0 0\\
0 0 0 2\\
0 1 0 2\\
0 2 0 2\\
}
&
\parbox[t]{0.5\textwidth}{\ttfamily
0 parked at (0, 2).\\
1 parked at (0, 1).\\
2 parked at (0, 3).
} \\
\hline
\end{longtable}

\section*{Sample 2}
\begin{longtable}[!h]{|p{0.5\textwidth}|p{0.5\textwidth}|}
\hline
\textbf{Input} & \textbf{Output} \\
\hline
\parbox[t]{0.5\textwidth}{\ttfamily
1 5 6\\
3\\
0 0 0 0 0\\
0 0 0 1\\
0 1 0 2\\
0 2 0 3\\
0 3 0 3\\
3 0 1\\
0 4 0 3\\
}
&
\parbox[t]{0.5\textwidth}{\ttfamily
0 parked at (0, 1).\\
1 parked at (0, 2).\\
2 parked at (0, 3).\\
3 parked at (0, 5/2).\\
Rearranged 1 bikes in 0.\\
4 parked at (0, 5/2).
} \\
\hline
\end{longtable}

\section*{Sample 3}
\begin{longtable}[!h]{|p{0.5\textwidth}|p{0.5\textwidth}|}
\hline
\textbf{Input} & \textbf{Output} \\
\hline
\parbox[t]{0.5\textwidth}{\ttfamily
1 5 6\\
3\\
3 4 5 6 7\\
0 0 0 1\\
0 1 0 1\\
0 2 0 1\\
0 3 0 1\\
2 0 1\\
4 6\\
}
&
\parbox[t]{0.5\textwidth}{\ttfamily
0 parked at (0, 1).\\
1 parked at (0, 2).\\
2 parked at (0, 3).\\
3 parked at (0, 3/2).\\
At 6, 3 bikes was fetched.
} \\
\hline
\end{longtable}

\section*{Sample 4}
\begin{longtable}[!h]{|p{0.5\textwidth}|p{0.5\textwidth}|}
\hline
\textbf{Input} & \textbf{Output} \\
\hline
\parbox[t]{0.5\textwidth}{\ttfamily
6 5 6\\
3 3 3 4 4 4\\
3 4 5 6 7\\
0 2 1\\
0 5 4\\
1 5 7\\
2 3 2\\
4 5 3\\
0 0 0 1\\
1 0 1 1\\
1 0 4 1\\
1 0 2 1\\
1 0 3 1\\
1 0 2 1\\
}
&
\parbox[t]{0.5\textwidth}{\ttfamily
0 parked at (0, 1).\\
0 moved to 1 in 11 seconds.\\
0 moved to 4 in 10 seconds.\\
0 moved to 2 in 8 seconds.\\
0 moved to 3 in 2 seconds.\\
0 moved to 2 in 2 seconds.
} \\
\hline
\end{longtable}
\newpage
\section*{Sample 5}
\begin{longtable}[!h]{|p{0.5\textwidth}|p{0.5\textwidth}|}
\hline
\textbf{Input} & \textbf{Output} \\
\hline
\parbox[t]{0.5\textwidth}{\ttfamily
5 10 10\\
5 10 6 11 2\\
0 0 0 0 0 0 0 0 0 0\\
3 0 49410\\
3 2 54898\\
2 1 76874\\
4 1 14829\\
0 6 4 1\\
5 3 0 315398\\
0 0 4 1\\
0 3 4 2\\
0 2 4 1\\
2 4 18337236\\
0 7 2 4\\
1 7 2 5\\
0 4 0 3\\
2 2 37134602\\
}
&
\parbox[t]{0.5\textwidth}{\ttfamily
6 parked at (4, 1).\\
0 parked at (4, 2).\\
3 parked at (4, 3/2).\\
2 parked at (4, 5/4).\\
7 parked at (2, 4).\\
7 moved to 2 in 0 seconds.\\
4 parked at (0, 3).
} \\
\hline
\end{longtable}



\section*{Subtasks}

在一個子任務的「測試資料範圍」的敘述中,如果存在沒有提到範圍的變數,則此變數的範圍為 Input 所描述的範圍。

\begin{center}
 \begin{tabular}{||c c c||} 
 \hline
 子任務編號 & 子任務配分 & 測試資料範圍 \\  
 \hline
 \hline
 1 & 10\% & $n \le 300, q \le 300$,僅 Park、Move \\ 
 \hline
 2 & 20\% & $n \le 300, q \le 300$,含 Fetch \\
 \hline 
 3 & 20\% & $n \le 300$,含 Fetch \\
 \hline
 4 & 50\% & 不含 Rebuild \\
 \hline
 5 & Bonus & 包含所有操作,完成可獲飲料獎勵 \\
 \hline
\end{tabular}
\end{center}



\end{document}
