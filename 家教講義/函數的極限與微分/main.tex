\documentclass[12pt]{article}
\usepackage{xeCJK}
\usepackage{fontspec}
\setCJKmainfont{Noto Serif CJK TC}[Script=CJK]
\usepackage{amsmath, amssymb, amsthm}
\usepackage{geometry}
\usepackage{tcolorbox}
\usepackage{fancyhdr}
\geometry{a4paper, margin=2.5cm}

\lhead{函數的極限與微分}
\rhead{極限、微分}

\begin{document}
\begin{tcolorbox}[colframe=black!60!blue, colback=blue!5!white, arc=4pt, boxrule=1pt]
\section{函數的極限}
\begin{enumerate}
    \item 函數的極限有分成\textbf{左極限}和\textbf{右極限}:
    \begin{itemize}
        \item 左極限:$\displaystyle\lim_{x \to a^-} f(x) = L$,表示當 $x$ 趨近於 $a$ 時,$f(x)$ 從左邊接近 $L$。
        \item 右極限:$\displaystyle\lim_{x \to a^+} f(x) = L$,表示當 $x$ 趨近於 $a$ 時,$f(x)$ 從右邊接近 $L$。
    \end{itemize}
    \item 如果左極限和右極限都存在且相等,則稱 $\displaystyle\lim_{x \to a} f(x) = L$。
    \item 如果 $\displaystyle\lim_{x \to a} f(x)$ 存在,且$\displaystyle f(a) = L$,則稱 $f(x)$ 在 $x=a$ 處\textbf{連續}。
    \item \textbf{勘根定理}:假設$f(x)$在$[a, b]$連續,且$f(a)f(b) < 0$,則存在$c \in (a, b)$使得$f(c) = 0$。
\end{enumerate}

\end{tcolorbox}

例題:
\begin{enumerate}
    \item 已知$f(x)$是三次實係數多項式且$k$為一個常數。若$\displaystyle\lim_{x\to 2}\frac{f(x)}{x-2}=-5$,
    $\displaystyle\lim_{x\to 3}\frac{f(x)}{x-3}=7$,且$\displaystyle\lim_{x\to 1}\frac{f(x)+k}{x-1}$存在,求
    $\displaystyle\lim_{x\to 1}\frac{f(x)+k}{x-1}$的值。\vspace{3cm}
    \item 設$P$點在拋物線$y=x^2$上且在第一象限,$O$為原點。在$x$軸正向上取一點$Q$,使的$\overline{OP}=\overline{OQ}$,
    接直線$\overline{QP}$交$y$軸於$R$點。當$P$點沿著拋物線趨近於原點時,$R$點的座標趨近於何值?\vspace{3cm}\newpage
    \item 設$f(x)=(x-16)^2(x-17)^2+3x$,請證明至少有一個時數$c$使得$f(c)=50$。\vspace{3cm}
    \item 承上題,已知其中一個$c$介於$16$與$17$之間,請求出$c$的值。\vspace{3cm}
    \item 請求出以下函數的極限:
    \begin{enumerate}
        \item $\displaystyle\lim_{x \to 6} (\frac{2x-17}{x^2-7x+6}+\frac{x-5}{x-6})$\vspace{2cm}
        \item $\displaystyle\lim_{x \to 1} \frac{1}{x-1}(\frac{1-x^{20}}{1-x}-20)$\vspace{2cm}
        \item $\displaystyle\lim_{x \to -1} \frac{|x^2-2x-3|}{x+1}$\vspace{2cm}
        \item $\displaystyle\lim_{x \to 0} \frac{|2+3x-x^2|-2}{x}$\vspace{2cm}
        \item $\displaystyle\lim_{x \to 0} x^2[\frac{1}{x}]$\vspace{2cm}
    \end{enumerate}
\end{enumerate}

\begin{tcolorbox}[colframe=black!60!blue, colback=blue!5!white, arc=4pt, boxrule=1pt]
\section{微分}
\begin{enumerate}
    \item 微分的概念是描述函數在某點的變化率。簡言之,我們討論當$x$在某點有極小的變化時,$f(x)$的變化量。微分也可以稱作為\textbf{導數}。
    \item 因此,導數的公式可以寫成:$$f'(a)=\displaystyle\lim_{x\to a}\cfrac{f(x)-f(a)}{x-a}$$,
    同時也能表現成$$f'(a) = \lim_{h \to 0} \frac{f(a+h) - f(a)}{h}$$,其中$h$是$x-a$的變化量。
    \item 如果說$f(x)$在某一點可微,則代表$f(x)$在該點\textbf{連續且極限存在}。
    \item \textbf{萊布尼茲符號}:導數也可以用萊布尼茲符號表示為$\frac{dy}{dx}$,其中$y=f(x)$。
    \item 導數的幾何意義是函數圖形在某點的切線斜率。
    \item 微分的計算規則:
    \begin{enumerate}
        \item 常數函數的導數為零:$f(x) = c \Rightarrow f'(x) = 0$。
        \item 幂函數的導數:$f(x) = x^n \Rightarrow f'(x) = nx^{n-1}$。
        \item 和差法則:$f(x) = g(x) + h(x) \Rightarrow f'(x) = g'(x) + h'(x)$。
        \item 乘法法則:$f(x) = g(x)h(x) \Rightarrow f'(x) = g'(x)h(x) + g(x)h'(x)$。
        \item 除法法則:$f(x) = \frac{g(x)}{h(x)} \Rightarrow f'(x) = \frac{g'(x)h(x) - g(x)h'(x)}{(h(x))^2}$。
        \item 鏈式法則:如果$y = g(u)$且$u = f(x)$,則$y' = g'(u)f'(x)$。
        \item 指數函數的導數:$f(x) = a^x \Rightarrow f'(x) = a^x \ln(a)$。
        \item 對數函數的導數:$f(x) = \log_a(x) \Rightarrow f'(x) = \frac{1}{x \ln(a)}$。
    \end{enumerate}
\end{enumerate}

\end{tcolorbox}

例題:
\begin{enumerate}
    \item 下列哪些函數在$x=0$處可微?
    \begin{enumerate}
        \item $f(x)=x+|x|$
        \item $f(x)=x|x|$
        \item $f(x)=x-[x]$
        \item $f(x)=\sqrt|x|$
        \item $f(x)=x^2\sin\frac{1}{x}, x\neq 0;0, x=0$
    \end{enumerate}
    \item 請計算下列函數的導函數:
    \begin{enumerate}
        \item $f(x) = 3x^2 - 5x + 2$\vspace{1.5cm}
        \item $f(x) = \sqrt{x^2+1}$\vspace{1.1cm}
        \item $f(x) = \frac{x^2 + 1}{x - 1}$\vspace{1.1cm}
        \item $f(x)=(2x+1)^{200}$\vspace{1.1cm}
        \item $f(x)=(2x+1)^{100}(2-3x)^{200}$\vspace{1.1cm}
    \end{enumerate}
    \item 已知兩個曲線$y=x^3+ax$和$y=x^2+bx+c$都通過點$P(1, 2)$,且他們在點$P$處的切線斜率相等。求$a, b, c$的值。\vspace{2.1cm}
    \item 設$f(x)=(x-a)(x-b)(x-c)$,其中$a, b, c$為實數,且$a>b>c$,$a+c>2b$。請排序$f'(a), f'(b), f'(c)$的大小。\vspace{2.1cm}
    \item 設$f(x)=(x^2-3)^3(2x-1)^2$,$g(x)=\frac{x(x+1)(x+2)(x+3)}{(x-1)(x-2)(x-3)}$,求$f'(1)+g'(0)$\vspace{2.1cm}
    \item 已知直線$x+y=2$和曲線$y=ax^3$相切,求$a$的值。\vspace{2cm}
    \item 設函數$f(x)=|(x-1)^3(x+1)|$,選出正確的選項:
    \begin{enumerate}
        \item $f(x)$在$x=1$處導數存在
        \item $f(x)$在$x=-1$可微分
        \item $f'(0)=-2$
        \item $\displaystyle\lim_{x\to 1}f(x)=f(1)$
        \item $f(x)$在$x=-1$處連續
    \end{enumerate}
\end{enumerate}

\end{document}
