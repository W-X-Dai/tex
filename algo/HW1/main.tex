\documentclass[12pt,a4paper]{article}
\usepackage[margin=2cm]{geometry}
\usepackage{xeCJK}
\usepackage{fontspec}
\setCJKmainfont{Noto Serif CJK TC}[Script=CJK]
\usepackage{amsmath,amssymb}
\usepackage{graphicx}
\usepackage{fancyhdr}
\setlength{\headheight}{14.5pt}
\addtolength{\topmargin}{-2.5pt}
\usepackage{hyperref}
\usepackage{listings}
\usepackage{enumitem}
\usepackage{titlesec}
\usepackage{caption}
\usepackage{indentfirst}
\usepackage{booktabs}
\usepackage{longtable}
\usepackage{multirow}
\usepackage{array}
\usepackage{tabularx}
\usepackage{float}
\usepackage{minted}
\setlength{\parindent}{2em}
\pagestyle{fancy}
\fancyhf{}
\cfoot{\thepage}
\linespread{1.3}
\setminted{
    linenos,                % 行號
    frame=lines,            % 上下框線
    framesep=5pt,           % 程式碼與邊框距離
    numbersep=8pt,          % 行號與程式碼距離
    fontsize=\scriptsize,   % 字體大小
    breaklines,             % 自動換行
    tabsize=4,              % tab 寬度
    rulecolor=\color{black},% 框線顏色
    xleftmargin=1.5em       % 左側縮排
}

\title{The Design and Analysis of Algorithms HW1}
\author{B12508026戴偉璿}
\date{\today}

\begin{document}

\maketitle

\lhead{The Design and Analysis of Algorithms HW1}
\rhead{B12508026戴偉璿}

\begin{enumerate}
    \item 
    \begin{enumerate}
        \item $\cfrac{n}{\log^3 n} = O(n)$, $64^{\sqrt{n}} = O(2^{\sqrt{6n}})$, $\log^{10} n = O(\log n)$, \\$\log(n!) = O(n \log n)$ (by Stirling's approximation), $n\sqrt n = O(n^{3/2})$, $\log 3^n = O(n)$, $n\log^2 n$
        \item
        \begin{enumerate}
            \item \boxed{\textbf{Disprove.}} To prove or disprove $n^\frac{1}{2} = O(n^\frac{1}{3})$, consider the definition of Big-O notation. $n^\frac{1}{2}\le cn^\frac{1}{3}$ where $c$ is a constant. Dividing both sides by $n^\frac{1}{3}$ gives $n^\frac{1}{6}\le c$. As $n$ approaches infinity, $n^\frac{1}{6}$ also approaches infinity, so there is no constant $c$ that satisfies the inequality. Therefore, $n^\frac{1}{2} \neq O(n^\frac{1}{3})$.
            \item \boxed{\textbf{Prove.}} To prove or disprove $3^n = \Omega(27^{\sqrt{n}})$, consider the definition of Big-Omega notation. $3^n\ge c27^{\sqrt{n}}$ where $c$ is a constant. Dividing both sides by $27^{\sqrt{n}}$ gives $\cfrac{3^n}{27^{\sqrt{n}}}=3^{n-3\sqrt{n}}\ge c$. As $n$ approaches infinity, $n-3\sqrt{n}$ also approaches infinity, so there exists a constant $c$ that satisfies the inequality. Therefore, $3^n = \Omega(27^{\sqrt{n}})$.
        \end{enumerate}
    \end{enumerate}
    \item
    \begin{enumerate}
        \item The outer loop runs from $1$ to $n$, and the inner loop runs from $1$ to $\sqrt{i}$. So the total number of iterations is:$\displaystyle\sum^n_{k=1}{\sqrt{k}}$. To analysis this, we can use the integral method:$\displaystyle\sum^n_{k=1}{\sqrt{k}} \approx \int_1^n \sqrt{x} \, dx = \left[ \frac{2}{3} x^{3/2} \right]_1^n = \frac{2}{3} (n^{3/2} - 1) = \Theta(n^{3/2})$.
    \end{enumerate}
\end{enumerate}

\end{document}