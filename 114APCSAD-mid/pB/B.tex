\documentclass[11pt,a4paper]{article}

\usepackage{indentfirst}
\usepackage{amssymb}
\usepackage{subcaption}
\usepackage{graphicx}
\usepackage{longtable}
\usepackage{fancyhdr}
\usepackage{xeCJK}
\usepackage{amsmath}
\usepackage{amssymb}
\usepackage{ulem}
\usepackage{xcolor}
\usepackage{fancyvrb}
\usepackage{listings}
\usepackage{soul}
\usepackage{hyperref}
	\lstdefinestyle{C}{
   		language=C, 
   		basicstyle=\ttfamily\bfseries,
    	numbers=left, 
    	numbersep=5pt,
    	tabsize=4,
    	frame=single,
   	 	commentstyle=\itshape\color{brown},
    	keywordstyle=\bfseries\color{blue},
   	 	deletekeywords={define},
    	morekeywords={NULL,bool}
	}	

\setCJKmainfont{Noto Serif CJK TC}[Script=CJK]
\setCJKmonofont{Noto Sans Mono CJK TC}[Script=CJK]
 
\voffset -20pt
\textwidth 410pt
\textheight 650pt
\oddsidemargin 20pt
\newcommand{\XOR}{\otimes}
\linespread{1.2}\selectfont

\pagestyle{fancy}
\lhead{114APCS進階班期中比賽}

\begin{document}

\begin{center}
\section*{B. Eval Ceremony}
\end{center}

\section*{Description}


萬惡的\textbf{蘿糕}!總是想著要取代\textbf{疑中}成為\textbf{歌瑪蘭共和國}內唯一的頂級中學,可是在資訊競賽方面卻屢屢受挫。
\textbf{蘿糕}每次派出去的選手都被優秀的\textbf{黃瓜}學長擊敗,好不容易捱到\textbf{黃瓜}畢業了,
沒想到\textbf{黃瓜}的傳人-\textbf{李岳岳}-竟然也能夠扛起大旗,將\textbf{蘿糕}的選手們踩在腳下。
為此,\textbf{蘿糕}決定要不擇手段召喚卓越大邪神AKA黑暗大髮絲....(以下省略)報復\textbf{疑中}!

但是卓越大邪神AKA黑暗大髮絲....(以下省略)要求很多,例如要獻祭睿璿學長的貼身馬克杯,或是稀有神獸日京元某處的毛髮。
而且他對獻祭日期也十分要求,閏年不能舉行、日期要不等於月份、月份也要不是日期的倍數、日期要非偶數且不等於13。
請幫蘿糕學生判斷那些日期能夠成功召喚卓越大邪神AKA黑暗大髮絲....(以下省略)擊潰疑中代表黃瓜的傳人-李岳岳,並贏得資訊競賽!

\section*{Input}

輸入第一行為一個整數$T(T\le 10^5)$,接下來有$T$筆詢問

接下來$T$行每行三個整數$Y,M,D$分別代表年(西元年),月,日

\section*{Output}

如果輸入日期不存在輸出\texttt{error}

若日期存在但不能舉行邪教儀式則輸出\texttt{No}(大寫須注意)

若日期存在且能舉行儀式則輸出\texttt{Yes}(大寫須注意)

\begin{itemize}
\item 根據維基百科,閏年規則如下:
    \begin{enumerate}
    \item 公元年分非4的倍數,為平年。
    \item 公元年分為4的倍數但非100的倍數,為閏年。
    \item 公元年分為100的倍數但非400的倍數,為平年。
    \item 公元年分為400的倍數為閏年。
    \end{enumerate}
\item 西元年從"一"開始記年
\item 保證所有的輸入皆是整數
\end{itemize}

\newpage

\section*{Sample 1}
\begin{longtable}[!h]{|p{0.5\textwidth}|p{0.5\textwidth}|}
\hline
\textbf {Input}	& \textbf {Output} \\
\hline
\parbox[t]{0.5\textwidth} % sample 1
{ \tt
3 \\
2000 2 29 \\
2003 7 11 \\
2021 4 32 \\
} &
\parbox[t]{0.5\textwidth}
{ \tt
%output
No\\
Yes\\
error\\

} \\
\hline
\end{longtable}


\section*{Subtasks}

在一個子任務的「測試資料範圍」的敘述中,如果存在沒有提到範圍的變數,則此變數的範圍為 Input 所描述的範圍。

\begin{center}
 \begin{tabular}{||c c c||} 
 \hline
 子任務編號 & 子任務配分 & 測試資料範圍 \\  
 \hline
 \hline
 1 & 20\% & 年份不為閏年且不會出現\texttt{error}的狀況 \\ 
 \hline
 2 & 30\% & 年份不為閏年\\
 \hline 
 3 & 50\% & 無限制 \\
 \hline

\end{tabular}
\end{center}



\end{document}
