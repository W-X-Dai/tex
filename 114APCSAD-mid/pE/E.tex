\documentclass[11pt,a4paper]{article}

\usepackage{indentfirst}
\usepackage{amssymb}
\usepackage{subcaption}
\usepackage{graphicx}
\usepackage{longtable}
\usepackage{fancyhdr}
\usepackage{xeCJK}
\usepackage{amsmath}
\usepackage{amssymb}
\usepackage{ulem}
\usepackage{xcolor}
\usepackage{fancyvrb}
\usepackage{listings}
\usepackage{soul}
\usepackage{hyperref}
	\lstdefinestyle{C}{
   		language=C, 
   		basicstyle=\ttfamily\bfseries,
    	numbers=left, 
    	numbersep=5pt,
    	tabsize=4,
    	frame=single,
   	 	commentstyle=\itshape\color{brown},
    	keywordstyle=\bfseries\color{blue},
   	 	deletekeywords={define},
    	morekeywords={NULL,bool}
	}	

\setCJKmainfont{Noto Serif CJK TC}[Script=CJK]
\setCJKmonofont{Noto Sans Mono CJK TC}[Script=CJK]
 
\voffset -20pt
\textwidth 410pt
\textheight 650pt
\oddsidemargin 20pt
\newcommand{\XOR}{\otimes}
\linespread{1.2}\selectfont

\pagestyle{fancy}
\lhead{114APCS進階班期中比賽}

\begin{document}

\begin{center}
\section*{E. Square Query}
\end{center}

\section*{Description}


你有 $n$ 個正方形,每個正方形的邊長分別是 $a_1, a_2, \ldots a_n$,以及 $q$ 次操作,每次操作為下列兩種其中一種:

\begin{itemize}
  \item $1\ l\ r\ k$:將第 $l$ 個到第 $r$ 個正方形的邊長加上 $k$
  \item $2\ l\ r$:計算第 $l$ 個到第 $r$ 個正方形的面積和取 $10^9+7$ 餘數的結果並輸出
\end{itemize}



\section*{Input}

第一行為一個正整數 $t$,代表子測資數量。

每組子測資的第一行會有兩個正整數 $n,q$,代表正方形的數量及操作次數。

接下來有 $n$ 個正整數 $a_1,a_2,\ldots,a_n(1\le a_i\le 10^9)$,為每個正方形的邊長。

接下來有 $q$ 行,每一行有一個正整數 $op$,若 $op=1$,則後面有 $3$ 個數字 $l, r, k$;若 $op=2$ 則後面有 $2$ 個數字 $l, r$。

保證所有子測資的 $n$ 的總和不超過 $25\times 10^4$,保證所有子測資的 $q$ 的總和不超過 $25\times 10^4$。

\textbf{測資範圍限制}
\begin{itemize}
  \item $1 \le t \le 10^4$
  \item $1\le n,q\le 25\times 10^4$
  \item $op \in \{1, 2\}$
  \item $1\le l\le r\le n$
  \item $1\le k\le 10^9$
\end{itemize}

\section*{Output}

對於操作 $2$ ,請輸出一行答案。

\newpage
\section*{Sample 1}
\begin{longtable}[!h]{|p{0.5\textwidth}|p{0.5\textwidth}|}
\hline
\textbf {Input}	& \textbf {Output} \\
\hline
\parbox[t]{0.5\textwidth} % sample 1
{ \tt
4\\
5 2\\
1 1 1 1 1\\
1 2 4 7\\
2 1 5\\
5 5\\
1 1 1 1 1\\
2 1 5\\
2 1 3\\
1 2 4 7\\
2 4 5\\
2 1 5\\
5 4\\
1 1 1 2 1\\
2 3 4\\
1 2 4 1\\
2 2 4\\
2 4 5\\
20 8\\
92 100 71 51 1 20 57 33 16 17 75 49 17 98 16 4 97 85 68 49\\
1 12 16 72\\
2 6 13\\
2 11 20\\
2 9 18\\
2 15 18\\
1 16 18 47\\
2 15 18\\
2 15 15\\
} &
\parbox[t]{0.5\textwidth}
{ \tt
%output
194\\
5\\
3\\
65\\
194\\
5\\
17\\
10\\
33470\\
94266\\
87786\\
30154\\
61033\\
7744\\
} \\
\hline
\end{longtable}


\section*{Subtasks}

在一個子任務的「測試資料範圍」的敘述中,如果存在沒有提到範圍的變數,則此變數的範圍為 Input 所描述的範圍。

\begin{center}
 \begin{tabular}{||c c c||} 
 \hline
 子任務編號 & 子任務配分 & 測試資料範圍 \\  
 \hline
 \hline
 1 & 0\% & 範例測資 \\ 
 \hline
 2 & 30\% & $n, q\le 250$\\
 \hline 
 3 & 70\% & 無其他限制 \\
 \hline

\end{tabular}
\end{center}
\end{document}
